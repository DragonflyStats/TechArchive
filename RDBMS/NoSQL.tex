\subsection{NoSQL}

NoSQL database, also called Not Only SQL, is an approach to data management and database design that's useful for very large sets of distributed data.  

NoSQL, which encompasses a wide range of technologies and architectures, seeks to solve the scalability and big data performance issues that relational databases weren’t designed to address. NoSQL is especially useful when an enterprise needs to access and analyze massive amounts of unstructured data or data that's stored remotely on multiple virtual servers in the cloud.   

Contrary to misconceptions caused by its name, NoSQL does not prohibit structured query language (SQL). While it's true that some NoSQL systems are entirely non-relational, others simply avoid selected relational functionality such as fixed table schemas and join operations. For example, instead of using tables, a NoSQL database might organize data into objects, key/value pairs or tuples.

\subsection{NoSQL implementations}
Arguably, the most popular NoSQL database is Apache Cassandra. Cassandra, which was once Facebook’s proprietary database, was released as open source in 2008. Other NoSQL implementations include SimpleDB, Google BigTable, Apache Hadoop, MapReduce, MemcacheDB, and Voldemort. Companies that use NoSQL include NetFlix, LinkedIn and Twitter.

NoSQL is often mentioned in conjunction with other big data tools such as massive parallel processing, columnar-based databases and Database-as-a-Service (DaaS).

\subsection{Background of NoSQL}
Relational databases were introduced into the 1970s to allow applications to store data through a standard data modeling and query language (Structured Query Language, or SQL). At the time, storage was expensive and data schemas were fairly simple and straightforward. Since the rise of the web, the volume of data stored about users, objects, products and events has exploded. Data is also accessed more frequently, and is processed more intensively – for example, social networks create hundreds of millions of customized, real-time activity feeds for users based on their connections' activities.

Even rendering a single web page or answering a single API request may take tens or hundreds of database requests as applications process increasingly complex information. Interactivity, large user networks, and more complex applications are all driving this trend.

In response to this demand, computing infrastructure and deployment strategies have also changed dramatically. Low-cost, commodity cloud hardware has emerged to replace vertical scaling on highly complex and expensive single-server deployments. And engineers now use agile development methods, which aim for continuous deployment and short development cycles, to allow for quick response to user demand for features.

\subsubsection{The Need for NoSQL}

Relational databases were never designed to cope with the scale and agility challenges that face modern applications – and aren't built to take advantage of cheap storage and processing power that's available today through the cloud. Relational database vendors have developed two main technical approaches to address these shortcomings:
