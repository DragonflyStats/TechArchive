Database management systems (DBMSs) and, in particular, relational DBMSs (RDBMSs)
are designed to do all of these things well. Their strengths are

- 1. To provide fast access to selected parts of large databases.
- 2. Powerful ways to summarize and cross-tabulate columns in databases.
- 3. Store data in more organized ways than the rectangular grid model of spreadsheets and R
data frames.
- 4. Concurrent access from multiple clients running on multiple hosts while enforcing security
constraints on access to the data.
- 5. Ability to act as a server to a wide range of clients.

<hr>
###  Normal Forms
- Third Normal Form
- Boyce Codd Third normal form

<hr>
Redundancy

Functional Relations

“One-to-One and Onto” : Surjective, bijective, injective.
<hr>
### SQL: structured query language

Commands:

-	Select
-	From 
-	Where

Logical operators

Open source platforms

Mysql

<hr>
\newpage
\section{Database Sciences}
 

Computer based support for decision making functions.
 
 
Data Collection and Data Problems
 
 
Sources of Data
            Internal
            External
Missing data
 
Dirty Data
 
 
Commercial Database services
 
IBM’s Information Warehouse
 
\section{Relational database}
 
DBMS (4.9)
The major role of the DBMS is to manage data
PostGreSQL
NoSQL
