
\section{SQL} 
SQL is a structured query language; used to access and manipulate data in databases
SQL is non-procedural and is very user friendly.
 
A simple query to indentify employees whose salary is greater than €20,000 can be written as

\begin{framed}
\begin{verbatim}
SELECT names, salaries
            FROM Employess
            WHERE Salary >20000
\end{verbatim}
\end{framed}

### SQL datatypes
http://philip.greenspun.com/sql/dates

\subsubsection{Data Type & Description }

integer(size),int(size),smallint(size),tinyint(size)     Hold integers only. The maximum number of digits  are  specified in parenthesis. 

decimal(size,d),numeric(size,d)  Hold numbers with fractions. The maximum number of digits are specified in "size". The maximum number of digits to the right of the decimal is specified in "d". 

char(size)  Holds a fixed length string (can contain letters, numbers, and special characters). The fixed size is specified in parenthesis. 

varchar(size)  Holds a variable length string (can contain letters, numbers, and special characters). The maximum size is specified in parenthesis. 

date(yyyymmdd)   Holds a date 

%=====================================================================%
\subsection{Triggers}
create or replace trigger insert_stock
before insert on stock for each row

\begin{framed}
\begin{verbatim}
declare 
price alert exception;
out_of_stock exceptio;
begin
  if (:new.unit_price < :new.unitcostprice) then
raise price_alert
endif
  if (:new.stock_level is null or : newstock_level<:new.reorder_level)
  then raise out_of_stock;
 end if;
exception
 when price alert then
  raise_application_error(-20002,'Below Cost selling');
 when out_of_stock then
  raise_applcation_error(-20001,'Need to reorder stock');
when others then rollback work;
end;
\end{verbatim}
\end{framed}

%==========================================================%
\subsection{AFTER TRIGGER }

create or replace trigger after_stock
after insert on stock for each row

\begin{framed}
\begin{verbatim}
begin
  if (:new.stock_level < :new.reorder_level) then
dbms_output.put_line(:new.stock_code||'being reordered.');
insert into restock values(sysdate,:new.stock_code);
endif
  if (:newunit_price<:new.unitcostprice)
  then 
dbms_output.put_line(:new.stock_code||'selling below cost _ record added.');
 end if;
exception
when others then 
 dbms_output.put_line('Unexpected error encountered - work rolled back');
 rollback work;
end;
\end{verbatim}
\end{framed}
%===============================================================%

\subsection*{Trigger data}

1. to show which triggers are active

\texttt{select triggername from user\_triggers}

2. to get more info

\begin{verbatim}
select trigger_name,trigger_type,triggering_event,table_name,referencing_names,trigger_body from user_triggers where trigger_name="<trigger_name>';
\end{verbatim}

3. to drop triggers
drop trigger<trigger_name>
