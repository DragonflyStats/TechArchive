Databases

Class2Go Jennifer Wisdom

reliability, efficiency, scalability, concurrency control, data abstractions, and high-level query languages
XML: Extensible Markup Language
Referential Integrity
Relational Database Management Systems (RDBMS)
Entity Relational Diagrams
NOSQL: Not Only SQL
Relational Algebra

----------------------------
Software Quick Guides

	Quick guide to XML validation and querying

	Quick guide to the RA relational algebra interpreter

	Quick guide to SQLite, MySQL, and PostgreSQL

----------------------------
Video Listings

1. Introduction
2. Relational Databases
3. XML data
4. JSON data
5. Relational Algebra
6. SQL
7. Relational Design Theory
8. Querying XML
9. Unified Modelling Language (UML)
10. Indexes
11. Constraints and Triggers
12. Transactions
13. Views
14. Authorization
15. Recursion in SQL
16. On-line analytical processing (OLAP)
17. NoSQL systems

--------------------------------
Relational Query Languages

Query languages:  Allow manipulation and retrieval 
of data from a database.

Relational model supports simple, powerful QLs:
 Strong formal foundation based on logic.
 Allows for much optimization.

Query Languages != programming languages!
 QLs not expected to be “Turing complete”.
 QLs not intended to be used for complex calculations.
 QLs support easy, efficient access to large data sets.

----------------------------

Formal Relational Query Languages

 Two mathematical Query Languages form 
the basis for “real” languages (e.g. SQL), and 
for implementation:

Relational Algebra:  More operational, very useful 
for representing execution plans.

Relational Calculus:   Lets users describe what they 
want, rather than how to compute it.  (Nonoperational,
 declarative.)

----------------------------

Preliminaries

A query is applied to relation instances, and the  result of a query is also a relation instance.
Schemas of input relations for a query are fixed (but query will run regardless of instance!)

The schema for the result of a given query is also fixed! Determined by definition of query language 
constructs.
   Positional vs. named-field notation:  
  Positional notation easier for formal definitions, 
named-field notation more readable.  
 Both used in SQL

----------------------------

Relational Algebra

Basic operations:
 Selection ( sigma )    Selects a subset of rows from relation.
 Projection ( pi    )   Deletes unwanted columns from relation.
 Cross-product (  times   )  Allows us to combine two relations.
 Set-difference (  -   ) Tuples in reln. 1, but not in reln. 2.
 Union (  U   ) Tuples in reln. 1 and in reln. 2.

Additional operations:
Intersection, join, division, renaming:  Not essential, but (very!) useful.
Since each operation returns a relation, operations can be composed! (Algebra is “closed”.)
----------------------------
Projection

Deletes attributes that are not in projection list.

Schema of result contains exactly the fields in the projection list, 
with the same names that they had in the (only) input relation.

Projection operator has to eliminate duplicates!  (Why??)

Note: real systems typically don’t do duplicate elimination unless the user explicitly asks 
for it.  (Why not?)

----------------------------
Equi-Join:  A special case of condition join where 
the condition c contains only equalities.

Result schema similar to cross-product, but only 
one copy of fields for which equality is specified.
 Natural Join: Equijoin on all common fields.

---------------------------
Summary
 The relational model has rigorously defined query languages that are simple and powerful.
 Relational algebra is more operational; useful as internal representation for query evaluation plans.
 Several ways of expressing a given query; a query optimizer should choose the most efficient version.
