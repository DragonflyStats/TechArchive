\documentclass{beamer}

\usepackage{amsmath}
\usepackage{amssymb}

\begin{document}

%--------------------------------------------------------%

\section{4 Formulas and Functions }

\begin{frame}
\Large
\textbf{Part 4. Formulas and Functions}
\begin{description}
\item[4.1] Arithmetic Formulas
\item[4.2] Functions
\end{description}
\end{frame}
%--------------------------------------------------------%
\begin{frame}
\frametitle{Arithmetic Formulas}
\Large
4.1.1 Recognize good practice in 
formula creation: refer to cell 
references rather than type 
numbers into formulas. 
\end{frame}
%--------------------------------------------------------%
\begin{frame}
\frametitle{Arithmetic Formulas}
\Large
4.1.2 Create formulas using cell 
references and arithmetic 
operators (addition, subtraction, 
multiplication, division). 
\end{frame}

%--------------------------------------------------------%
\begin{frame}[fragile]
\frametitle{Arithmetic Formulas}
\Large
 4.1.3 Identify and understand standard 
error values associated with 
using formulas: 
\begin{verbatim}
#NAME?
#DIV/0!
#REF! 
\end{verbatim}
\end{frame}
%--------------------------------------------------------%
\begin{frame}
\frametitle{Arithmetic Formulas}
\Large
 4.1.4 Understand and use relative, 
absolute cell referencing in 
formulas. 

\end{frame}

%Section 4.2
\section{4.2 Functions}
%--------------------------------------------------------%
\begin{frame}
\frametitle{4.2 Functions}
 4.2.1 Use sum, average, minimum, 
maximum, count, counta, round 
functions.
\end{frame}
%--------------------------------------------------------%
\begin{frame}
\frametitle{4.2 Functions}
4.2.2 Use the logical function if 
(yielding one of two specific 
values) with comparison 
operator: =, >, <. 
\end{frame}
\end{document}