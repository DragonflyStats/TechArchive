Definition:
A spreadsheet application is a computer program such as Excel, OpenOffice Calc, or Google Docs Spreadsheets.

It has a number of built in features and tools, such as functions, formulas, charts, and data analysis tools that make it easier to work with large amounts of data.

The term is also used to refer to the computer file created by the above mentioned programs.

In this sense, a spreadsheet is a file used to store various types of data.

The basic storage unit for a spreadsheet file is a table.

In a table, the data is arranged in rows and columns to make it easier to store, organize, and analyze the information.

In Excel an individual spreadsheet file is referred to as a workbook.

A term related to this is worksheet, which, in Excel, refers to a single page or sheet in a workbook. By default, Excel has three worksheets per workbook.

So, to put it all together, a spreadsheet program, such as Excel, is used to create workbook files that contain one or more worksheets containing data.

In Excel, a cell reference identifies the location a cell or group of cells in the spreadsheet. Sometimes referred to as a cell address, a cell reference consists of the column letter and row number that intersect at the cell's location. Note that when listing a cell reference, the column letter is always listed first.

Cell references are used in formulas, functions, charts, and other Excel commands. 

While references often refer to individual cells such as A1, B38, or Z345, they can also refer to a group or range of cells.

Ranges are identified by the cell references of the cells in the upper left and lower right corners of the range. 
The two cell references used for a range are separated by a colon ( : ) which tells Excel to include all the cells between these start and end points. 

An example of a range of adjacent cells would be B5:D10


%--------------------------------------------------------------------------------%
On completion of this module the candidate will be able to:
Work with spreadsheets and save them in different file formats
Choose built-in options, such as the Help function, within the application to enhance productivity
Enter data into cells; use good practice in creating lists
Select, sort and copy, move and delete data
Edit rows and columns in a worksheet
Copy, move, delete, and appropriately rename worksheets
Create mathematical and logical formulas using standard spreadsheet functions; use good practice in formula creation; recognise error values in formulas
Format numbers and text content in a spreadsheet
Choose, create, and format charts to communicate information meaningfully
Adjust spreadsheet page settings
Check and correct spreadsheet content before finally printing spreadsheets
%----------------------------------------------------------------------------------------%
1 Using the Application 
1.1 Working with Spreadsheets 1.1.1 Open, close a spreadsheet 
application. Open, close spreadsheets. 
 1.1.2 Create a new spreadsheet based 
on default template. 
 1.1.3 Save a spreadsheet to a location on a drive. Save a spreadsheet 
under another name to a location on a drive. 
 1.1.4 Save a spreadsheet as another file type like: template, text file, 
software specific file extension, version number. 
 1.1.5 Switch between open spreadsheets. 

1.2 Enhancing Productivity 
1.2.1 Set basic options/preferences in 
the application: user name, default folder to open, save  spreadsheets. 
 
 1.2.2 Use available Help functions

1.2.3 Use magnification/zoom tools. 
 1.2.4 Display, hide built-in toolbars. 
Restore, minimize the ribbon. 



%http://excelribbon.tips.net/T009093_Zooming_In_On_Your_Worksheet.html
%----------------------------------------------------------------------------------------%
2.1 Insert, Select 2.1.1 Understand that a cell in a 
worksheet should contain only 
one element of data, (for 
example, first name detail in one 
cell, surname detail in adjacent 
cell). 
 2.1.2 Recognize good practice in 
creating lists: avoid blank rows 
and columns in the main body of 
list, insert blank row before Total 
row, ensure cells bordering list 
are blank. 
 2.1.3 Enter a number, date, text in a 
cell. 
 2.1.4 Select a cell, range of adjacent 
cells, range of non-adjacent cells, 
entire worksheet. 
 2.2 Edit, Sort 2.2.1 Edit cell content, modify existing 
cell content. 
 2.2.2 Use the undo, redo command. 
 2.2.3 Use the search command for 
specific content in a worksheet. 
 2.2.4 Use the replace command for 
specific content in a worksheet. 
 2.2.5 Sort a cell range by one criterion 
in ascending, descending 
numeric order, ascending, 
descending alphabetic order. 
%---------------------------------%

2.3 Copy, Move, Delete 
2.3.1 Copy the content of a cell, cell  range within a worksheet, 
between worksheets, between open spreadsheets. 
 2.3.2 Use the autofill tool/copy handle 
tool to copy, increment data 
entries. 
 2.3.3 Move the content of a cell, cell 
range within a worksheet, 
between worksheets, between 
open spreadsheets. 
2.3.4 Delete cell contents. 


%----------------------------------------------------------------------------------------------%
3.2 Worksheets 3.2.1 Switch between worksheets. 
 3.2.2 Insert a new worksheet, delete a 
worksheet. 
 3.2.3 Recognize good practice in 
naming worksheets: use 
meaningful worksheet names 
rather than accept default 
names. 
 3.2.4 Copy, move, rename a 
worksheet within a spreadsheet. 



3.2 Worksheets 3.2.1 Switch between worksheets. 
 3.2.2 Insert a new worksheet, delete a 
worksheet. 
 3.2.3 Recognize good practice in 
naming worksheets: use 
meaningful worksheet names 
rather than accept default 
names. 
 3.2.4 Copy, move, rename a 
worksheet within a spreadsheet. 

%-------------------------------------------------------------------------------%

4 Formulas and 
Functions 
4.1 Arithmetic Formulas 4.1.1 Recognize good practice in 
formula creation: refer to cell 
references rather than type 
numbers into formulas. 
 4.1.2 Create formulas using cell 
references and arithmetic 
operators (addition, subtraction, 
multiplication, division). 
 4.1.3 Identify and understand standard error values associated with 
using formulas: #NAME?, 
#DIV/0!, #REF!. 
 4.1.4 Understand and use relative, absolute cell referencing in 
formulas. 
 4.2 Functions 4.2.1 Use sum, average, minimum, 
maximum, count, counta, round 
functions.
4.2.2 Use the logical function if 
(yielding one of two specific 
values) with comparison 
operator: =, >, <. 



In Excel and other spreadsheets, a relative cell reference identifies the location of a cell or group of cells.

Cell references are used in formulas, functions, charts , and other Excel commands.

By default, a spreadsheet cell reference is relative. What this means is that as a formula or function is copied and pasted to other cells, the cell references in the formula or function change to reflect the function's new location.

In contrast, an absolute cell reference does not change when it a formula is copied and pasted to other cells.

A relative cell reference consists of the column letter and row number that intersect at the cell's location.

An example of a relative cell reference would be C4, G15, or Z2345.

Note: When listing a cell reference - either relative or absolute, the column letter is always listed first.

Definition:
In Excel and other spreadsheets, an absolute cell reference identifies the location a cell or group of cells.

Cell references are used in formulas, functions, charts, and other Excel commands.

An absolute cell reference consists of the column letter and row number preceded by dollar signs ( $ ).

An example of an absolute cell reference would be $C$4, $G$15, or $A$345.

Note: An easy way to add the dollar signs to a cell reference is to click on a cell reference and then press the F4 key on the keyboard.

An absolute cell reference is used when you want a cell reference to stay fixed on a specific cell.

This means that as a formula or function is copied and pasted to other cells, the cell references in the formula or function do not change.

By contrast, most cell references in a spreadsheet are relative cell references, which change when copied and pasted to other cells.

More information can be found under Cell Reference and Relative Cell Reference. 



%--------------------------------------------------------------------%
5 Formatting 5.1 Numbers/Dates 5.1.1 Format cells to display numbers 
to a specific number of decimal 
places, to display numbers with, 
without a separator to indicate 
thousands. 
 5.1.2 Format cells to display a date 
style, to display a currency 
symbol. 
 5.1.3 Format cells to display numbers 
as percentages. 
 5.2 Contents 5.2.1 Change cell content appearance: 
font sizes, font types. 
 5.2.2 Apply formatting to cell contents: 
bold, italic, underline, double 
underline. 
 5.2.3 Apply different colours to cell 
content, cell background. 
 5.2.4 Copy the formatting from a cell, 
cell range to another cell, cell 
range. 
 5.3 Alignment, Border Effects 5.3.1 Apply text wrapping to contents 
within a cell, cell range. 
 5.3.2 Align cell contents: horizontally, 
vertically. Adjust cell content 
orientation. 
 5.3.3 Merge cells and centre a title in a 
merged cell. 
 5.3.4 Add border effects to a cell, cell 
range: lines, colours. 

%----------------------------------------------------------------%
6 Charts 6.1 Create 6.1.1 Create different types of charts 
from spreadsheet data: column 
chart, bar chart, line chart, pie 
chart. 
 6.1.2 Select a chart. 
 6.1.3 Change the chart type. 
 6.1.4 Move, resize, delete a chart. 
6.2 Edit 6.2.1 Add, remove, edit a chart title. 
 6.2.2 Add data labels to a chart: 
values/numbers, percentages. 
 6.2.3 Change chart area background 
colour, legend fill colour. 
 6.2.4 Change the column, bar, line, pie 
slice colours in the chart. 
 6.2.5 Change font size and colour of 
chart title, chart axes, chart 
legend text. 

%---------------------------------------------------------------%


7 Prepare Outputs 7.1 Setup 7.1.1 Change worksheet margins: top, 
bottom, left, right. 
 7.1.2 Change worksheet orientation: 
portrait, landscape. Change 
paper size. 
 7.1.3 Adjust page setup to fit 
worksheet contents on a 
specified number of pages. 
 7.1.4 Add, edit, delete text in headers, 
footers in a worksheet. 
 7.1.5 Insert and delete fields: page 
numbering information, date, 
time, file name, worksheet name 
into headers, footers. 
 7.2 Check and Print 7.2.1 Check and correct spreadsheet 
calculations and text. 
 7.2.2 Turn on, off display of gridlines, 
display of row and column 
headings for printing purposes. 
 7.2.3 Apply automatic title row(s) 
printing on every page of a 
printed worksheet. 
 7.2.4 Preview a worksheet. 
 7.2.5 Print a selected cell range from a 
worksheet, an entire worksheet, 
number of copies of a worksheet, 
the entire spreadsheet, a 
selected chart. 



On completion of this module the candidate will be able to:
Apply advanced formatting options such as conditional formatting and customised number formatting and handle worksheets
Use functions such as those associated with logical, statistical, financial and mathematical operations
Create charts and apply advanced chart formatting features
Work with tables and lists to analyse, filter and sort data. Create and use scenarios
Validate and audit spreadsheet data
Enhance productivity by working with named cell ranges, macros and templates
Use linking, embedding and importing features to integrate data
Collaborate on and review spreadsheets. Apply spreadsheet security features
- See more at: http://www.ecdl.org/programmes/index.jsp?p=2929&n=2951#sthash.r5yVBek4.dpuf

AM4.1 Formatting AM4.1.1 Cells AM4.1.1.1 Apply an autoformat/table style to a 
cell range. 
 AM4.1.1.2 Apply conditional formatting based on 
cell content. 
 AM4.1.1.3 Create and apply custom number 
formats. 
 AM4.1.2 Worksheets AM4.1.2.1 Copy, move worksheets between 
spreadsheets. 
 AM4.1.2.2 Split a window. Move, remove split 
bars. 
 AM4.1.2.3 Hide, show rows, columns, 
worksheets. 


AM4.4.1.4 Automatically, manually group data in 
a pivot table/datapilot and rename 
groups. 
 AM4.4.1.5 Use one-input, two-input data 
tables/multiple operations tables. 
 AM4.4.2 Sorting and 
Filtering 
 AM4.4.2.1 Sort data by multiple columns at the 
same time. 
 AM4.4.2.2 Create a customized list and perform 
a custom sort. 
 AM4.4.2.3 Automatically filter a list in place. 
 AM4.4.2.4 Apply advanced filter options to a list. 
 AM4.4.2.5 Use automatic sub-totalling features. 
 AM4.4.2.6 Expand, collapse outline detail levels. 
 AM4.4.3 Scenarios AM4.4.3.1 Create named scenarios. 
 AM4.4.3.2 Show, edit, delete scenarios. 
 AM4.4.3.3 Create a scenario summary report. 
AM4.5 Validating and 
Auditing 
 AM4.5.1 Validating AM4.5.1.1 Set, edit validation criteria for data 
entry in a cell range like: whole 
number, decimal, list, date, time. 
 AM4.5.1.2 Enter input message and error alert. 
 AM4.5.2 Auditing AM4.5.2.1 Trace precedent, dependent cells. 
Identify cells with missing 
dependents. 
 AM4.5.2.2 Show all formulas in a worksheet, 
rather than the resulting values. 
 AM4.5.2.3 Insert, edit, delete, show, hide 
comments/notes. 
AM4.6 Enhancing 
Productivity 
 AM4.6.1 Naming Cells AM4.6.1.1 Name cell ranges, delete names for 
cell ranges. 
 AM4.6.1.2 Use named cell ranges in a function. 
 AM4.6.2 Paste Special AM4.6.2.1 Use paste special options: add, 
subtract, multiply, divide. 
 AM4.6.2.2 Use paste special options: values 
/numbers, transpose. 
 AM4.6.3 Templates AM4.6.3.1 Create a spreadsheet based on an 
existing template. 

AM4.6.3.2 Modify a template. 
 AM4.6.4 Linking, 
Embedding and 
Importing 
 AM4.6.4.1 Insert, edit, remove a hyperlink. 
AM4.6.4.2 Link data within a spreadsheet, 
between spreadsheets, between 
applications. 
 AM4.6.4.3 Update, break a link. 
 AM4.6.4.4 Import delimited data from a text file. 
 AM4.6.5 Automation AM4.6.5.1 Record a simple macro like: change 
page setup, apply a custom number 
format, apply autoformats to a cell 
range, insert fields in worksheet 
header, footer. 
 AM4.6.5.2 Run a macro. 
 AM4.6.5.3 Assign a macro to a custom button 
on a toolbar. 
AM4.7 Collaborative 
Editing 
 AM4.7.1 Tracking and 
Reviewing 
 AM4.7.1.1 Turn on, off track changes. Track 
changes in a worksheet using a 
specified display view. 
 AM4.7.1.2 Accept, reject changes in a 
worksheet. 
 AM4.7.1.3 Compare and merge spreadsheets. 
 AM4.7.2 Security AM4.7.2.1 Add, remove password protection for 
a spreadsheet: to open, to modify. 
 AM4.7.2.2 Protect, unprotect cells, worksheet 
with a password. 
 AM4.7.2.3 Hide, unhide formulas.
