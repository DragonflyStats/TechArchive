
\documentclass[a4paper,11pt]{report}
%
%--------------------   start of the 'preamble'
%
\usepackage{graphicx,amssymb,amstext,amsmath}

\usepackage[margin=0.40in]{geometry}


%%    homebrew commands -- to save typing
\newcommand\etc{\textsl{etc}}
\newcommand\eg{\textsl{eg.}\ }
\newcommand\etal{\textsl{et al.}}
\newcommand\Quote[1]{\lq\textsl{#1}\rq}
\newcommand\fr[2]{{\textstyle\frac{#1}{#2}}}
\newcommand\miktex{\textsl{MikTeX}}
\newcommand\comp{\textsl{The Companion}}
\newcommand\nss{\textsl{Not so Short}}
%
%---------------------   end of the 'preamble'
%

\begin{document}
\chapter{Neural Networks}
\section{Overview}
\begin{itemize}
\item The artificial neuron; \item network architecture;
perceptrons. \item Single layer networks; supervised training in
batch and individual mode. \item Multilayer feedforward networks;
\item backpropogation; momentum. \item Counterpropogation
networks; unsupervised training; initialisation of weights. \item
Statistical methods; Boltzmann training. \item Feedback networks;
\item Hopfields nets; energy; training. Applications. \item
Additional software requirements: recommended that some neural
nets software is obtained (eg MATLAB).
\end{itemize}


\section{Hopfield networks}
A Hopfield network is a form of recurrent artificial neural
network invented by John Hopfield. Hopfield networks serve as
content-addressable memory systems with binary threshold units.
They are guaranteed to converge to a local minimum, but
convergence to one of the stored patterns is not guaranteed.


\section{Lyapunov functions}
Lyapunov functions are functions which can be used to prove the
stability of a certain fixed point in a dynamical system or
autonomous differential equation. Named after the Russian
mathematician Aleksandr Mikhailovich Lyapunov, Lyapunov functions
are important to stability theory and control theory. A similar
concept appears in the theory of general state space Markov
Chains, usually under the name Lyapunov-Foster functions.

Functions which might prove the stability of some equilibrium are
called Lyapunov-candidate-functions. There is no general method to
construct or find a Lyapunov-candidate-function which proves the
stability of an equilibrium, and the inability to find a Lyapunov
function is inconclusive with respect to stability, which means,
that not finding a Lyapunov function doesn't mean that the system
is unstable. For dynamical systems (e.g. physical systems),
conservation laws can often be used to construct a
Lyapunov-candidate-function.

The basic Lyapunov theorems for autonomous systems which are
directly related to Lyapunov (candidate) functions are a useful
tool to prove the stability of an equilibrium of an autonomous
dynamical system. One must be aware that the basic Lyapunov
Theorems for autonomous systems are a sufficient, but not
necessary tool to prove the stability of an equilibrium. Finding a
Lyapunov Function for a certain equilibrium might be a matter of
luck. Trial and error is the method to apply, when testing
Lyapunov-candidate-functions on some equilibrium. As the areas of
equal stability often follow lines in 2D, the computer generated
images of Lyapunov exponents are visually appealing and very
popular.

\section{Boltzmann Machines}
A Boltzmann machine is the name given to a type of stochastic
recurrent neural network by Geoffrey Hinton and Terry
Sejnowski.They are named after the Boltzmann distribution in
statistical mechanics, which is used in their sampling function.
Boltzmann machines can be seen as the stochastic, generative
counterpart of Hopfield nets. They were one of the first examples
of a neural network capable of learning internal representations,
and are able to represent and (given sufficient time) solve
difficult combinatoric problems.

Boltzmann machines with unconstrained connectivity have not proven
useful for practical problems in machine learning or inference.
They are still theoretically intriguing, however, due to the
locality and Hebbian nature of their training algorithm, as well
as their parallelism and the resemblance of their dynamics to
simple physical processes. If the connectivity is constrained, the
learning can be made efficient enough to be useful for practical
problems.

\section{Self Organizing Maps}
A self-organizing map (SOM) or self-organizing feature map (SOFM)
is a type of artificial neural network that is trained using
unsupervised learning to produce a low-dimensional (typically
two-dimensional), discretized representation of the input space of
the training samples, called a map. Self-organizing maps are
different from other artificial neural networks in the sense that
they use a neighborhood function to preserve the topological
properties of the input space.

This makes SOM useful for visualizing low-dimensional views of
high-dimensional data, akin to multidimensional scaling. The model
was first described as an artificial neural network by the Finnish
professor Teuvo Kohonen, and is sometimes called a Kohonen map.
[1] Like most artificial neural networks, SOMs operate in two
modes: training and mapping. Training builds the map using input
examples. It is a competitive process, also called vector
quantization. Mapping automatically classifies a new input vector.
A self-organizing map consists of components called nodes or
neurons. Associated with each node is a weight vector of the same
dimension as the input data vectors and a position in the map
space. The usual arrangement of nodes is a regular spacing in a
hexagonal or rectangular grid. The self-organizing map describes a
mapping from a higher dimensional input space to a lower
dimensional map space.

The procedure for placing a vector from data space onto the map is
to find the node with the closest weight vector to the vector
taken from data space and to assign the map coordinates of this
node to our vector. While it is typical to consider this type of
network structure as related to feedforward networks where the
nodes are visualized as being attached, this type of architecture
is fundamentally different in arrangement and motivation. Useful
extensions include using toroidal grids where opposite edges are
connected and using large numbers of nodes. It has been shown that
while self-organizing maps with a small number of nodes behave in
a way that is similar to K-means, larger self-organizing maps
rearrange data in a way that is fundamentally topological in
character. It is also common to use the U-matrix. The U-matrix
value of a particular node is the average distance between the
node and its closest neighbors (ref. 9). In a square grid for
instance, we might consider the closest 4 or 8 nodes, or six nodes
in a hexagonal grid. Large SOMs display properties which are
emergent. In maps consisting of thousands of nodes, it is possible
to perform cluster operations on the map itself.
%-----------------------------------------------------------------------------------%
\newpage
\section{Widrow - Hoff Rule}

\subsection{Least mean squares} Least mean squares (LMS) algorithms
are a class of adaptive filter used to mimic a desired filter by
finding the filter coefficients that relate to producing the least
mean squares of the error signal (difference between the desired
and the actual signal). It is a stochastic gradient descent method
in that the filter is only adapted based on the error at the
current time. It was invented in 1960 by Stanford University
professor Bernard Widrow and his first Ph.D. student, Ted Hoff.

%-----------------------------------------------------------------------------------%
\newpage
\section{Hebbian theory}
Hebbian theory describes a basic mechanism for synaptic plasticity
where in an increase in synaptic efficacy arises from the
presynaptic cell's repeated and persistent stimulation of the
postsynaptic cell. Introduced by Donald Hebb in 1949, it is also
called Hebb's rule, Hebb's postulate, and cell assembly theory,
and states: \begin{quote}Let us assume that the persistence or
repetition of a reverberatory activity (or "trace") tends to
induce lasting cellular changes that add to its stability.… When
an axon of cell A is near enough to excite a cell B and repeatedly
or persistently takes part in firing it, some growth process or
metabolic change takes place in one or both cells such that A's
efficiency, as one of the cells firing B, is increased. The theory
is often summarized as "cells that fire together, wire together"
\end{quote} although this is an oversimplification of the nervous system
not to be taken literally, as well as not accurately representing
Hebb's original statement on cell connectivity strength changes.
The theory is commonly evoked to explain some types of associative
learning in which simultaneous activation of cells leads to
pronounced increases in synaptic strength. Such learning is known
as Hebbian learning

%-----------------------------------------------------------------------------------%
\newpage
\section{Energy Well}
\begin{itemize}
\item Explain which neural network state is known as the `energy
well'. \item Write the formula for computing the energy level in a
Hopfield neural network.
\end{itemize}

\newpage
%http://www.cs.miami.edu/~geoff/Courses/COMP6210-10M/Content/NeuralNetworks.shtml
\begin{verbatim}

Neural Networks

--------------------------------------------------------------------------------
 • Processing Elements ◦ An array of inputs, with a weight associated with each input. 
◦ An intermediate value computed from the input values and weights. The computation performed depends on the architecture of the network, often the sum of products. 
◦ An output which is some (activation) function of the intermediate value. Often sigmoid, or step, or sign function. 
◦ Inputs come either from other PEs, or from an external source. 
◦ Layers or slabs ◾ The PEs are arranged into layers, or slabs. 
◾ The input layer accepts input from an external source 
◾ The output layer produces output for external consumption 
◾ Hidden layers lie between the input and output layers 
◾ There may or may not be a spatial relationship between the PEs in a layer. 
◾ If each PE in a layer has an input from every PE in the previous layer, then the network is fully feedforward connected. 
◾ Connections between non-adjacent layers may be used. 

◦ Learning ◾ Supervised learning architectures ◾ Learn functions from input to output, based on example input and output vector pairs. 
◾ Once trained can approximate the function for inputs not previously encountered. 

◾ Unsupervised learning architectures ◾ Learn to differentiate between different input vectors. 





• Kohonen/Counter-Propagation Networks ◦ There is an input layer, a hidden layer called the Kohonen layer, and an output layer called the Grossberg layer. 
◦ The network is fully feedforward connected. 
◦ Training takes place in two stages. ◾ Firstly the Kohonen layer is trained in an unsupervised manner. ◾ This trains the PEs in the layer to differentiate between different input vectors. 

◾ The second phase trains the Grossberg layer in a supervised manner. ◾ This trains the Grossberg layer to associate an output vector with each recognised input vector. 


◦ Once trained, the network will output an appropriate output vector for any given input vector. 
◦ Kohonen layer ◾ The PEs in the Kohonen layer may have a spatial relationship, e.g. rectangular lattice, triangular lattice, hexagonal lattice. 
◾ Intermediate value = sqrt(SUMinputs(Weight - Input)2)
 This is the Euclidian distance from the weight vector to the input vector. 
◾ Activation function is f(Intermediate) = 1 if Intermediate is minimum over PEs,
 f(Intermediate) = 0 otherwise
 This is a form of step function. 
◾ The weights are initially set randomly. 

◦ Training the Kohonen Layer ◾ A sequence of typical input vectors are handed to the input layer, which distributes these values to the Kohonen layer PEs. 
◾ Each Kohonen layer PE works out the intermediate value (distance) from its weights to the input vector. 
◾ The activation function determines which PE which is closest to the input, and is declared the "winner". 
◾ The winning PE (and those close by in the spatial relationship if any) have their weights moved towards the input vector:
 weightkohonen,winner#,input# += alpha*(inputkohonen,winner#,input# - weightkohonen,winner#,input#)
 where alpha is the learning rate for winners. 
◾ The other PEs (losers) have their weights moved towards the input vector:
 weightkohonen,loser#,input# += beta*(inputkohonen,loser#,input# - weightkohonen,loser#,input#)
 where beta is the learning rate for losers. Beta is typically very small, often 0. 
◾ After sufficient training, the PE's weights will be distributed about the input space, with higher density in the areas of frequent input. Each PE recognises a piece of the input space. 

◦ The Grossberg layer ◾ Intermediate value = SUMinputs(Weight * Input). This is the sum of products. 
◾ Activation function is f(Intermediate) = Intermediate 
◾ The weights are initially set randomly. 

◦ Training the Grossberg layer ◾ A sequence of typical input vectors and expected output vectors are used. 
◾ The input values are distributed to the Kohonen layer as usual, and the winner calulated. 
◾ The output values from the Kohonen layer are transmitted to the Grossberg layer. 
◾ The output from each Grossberg layer PE is the weight corresponding to the input value 1 from the Kohonen layer. 
◾ The weights in Grossberg layer PEs are then modified:
 weightgrossberg,pe#,input# += inputgrossberg,pe#,input# * (gamma*(expected_outputpe# - outputgrossberg,pe#,input#))
 where gamma is the learning rate for the Grossberg layer. 
◾ After sufficient training, the vector made from the Grossberg layer PE's outputs will approximate the output vector for the piece of input space recognised by the ith Kohonen layer PE. 

◦ Example application: Weather types and intelligent reactions 



• Backpropagation ◦ There is an input layer, a hidden layer and an output layer. 
◦ Intermediate value = SUMinputs(Weight * Input). This is the sum of products. 
◦ Activation function is the sigmoid: f(x) = 1/(1 + exp(-x)) 
◦ The network may or may not be fully feedforward connected. 
◦ The weights are initially set randomly. 
◦ The network is trained in a supervised manner, to learn a function from input vectors to output vectors (the output vectors are formed from the outputs from the output layer PEs). This is done by adjusting the weights in the network. 
◦ Once trained, the network is used by presenting an input vector. The vector formed from the output PE output forms the output from the function learned. 
◦ Training ◾ A sequence of typical input and output vector pairs are presented to the network, and the PE outputs calculated. 
◾ The output layer weights are then updated:
 deltaoutput,pe = f'(Intermediateoutput,pe) * (expected_outputoutput,pe - outputoutput,pe)
 weightoutput,pe,i += alpha*deltaoutput,pe*inputoutput,pe,i
 where the first derivative of the sigmoid function is used to train the network: f'(x) = f(x)*(1-f(x)), and alpha is the learning rate for the output layer. 
◾ The hidden layer weights are then updated, by propagating the errors from the output layer back to the hidden layer:
 deltahidden,pe = f'(Intermediatehidden,pe) * SUMi(deltaoutput,i*weightoutput,pe,i)
 weighthidden,pe,i += beta*deltahidden,pe*inputhidden,i
 where beta is the learning rate for the hidden layer. 

◦ Example application: Stock market analysis 




--------------------------------------------------------------------------------
 
Exam Style Questions
• Describe the structure and operation of a processing element. 
• Explain what is meant by a neural network being "fully feedforward connected". 
• Explain the difference between supervised and unsupervised learning in a neural network. 
• Describe the architecture of a Kohonen/counter propagation network. What (in general terms) are the tasks of each of the layers? 
• Give details of how the output and hidden layers of a backpropogation network are trained. After sufficient training, what has a backpropogation network learned? 

\end{verbatim}

\section{2010 Neural Networks Exam ULCIS}

\subsection{Question 1}
\begin{itemize}
\item[a)] Draw a diagram showing a threshold unit with 2 inputs (?a, ?b) with weights (3 and l
respectively) plus a bias with weight 2. Explain the meaning of the symbols that you
use.
[4]
\item[b)] What is meant by the terms recurrent, sigmoid, net and supervised learning? Use
diagrams to help your explanation if appropriate.
[8]
\item[c)] Describe the calculation involved in calculating the activation of a tmit given its
inputs.
[4]
\item[d)] A threshold unit has 5 inputs, and outputs a l if and only if more than 3 of these
inputs are l.
Describe the design of such a unit giving the values of all of the weights needed.
Explain how this unit works, showing that it obeys the description given.
[9]
\end{itemize}
%----------------------------------------------------------------------%

\subsection{Question 2}
\begin{itemize}
\item[a)]  A threshold unit has a bias and 3 inputs. The associated weights are 1, 1, -0.25 and
-0.2 respectively. Use this information to draw a diagram of the unit.
[5]
\item[b)] Draw a truth table for the Lmit with the given weights.
[5]
\item[c)] Describe the update rule for such a unit, explaining carefully any equations and
symbols that you use.
[5]
\item[d)] Showing all your calculations, calculate the effect on the unit of training with the
following training set and with a leaming rate of 0.2:
%?a ?b ?c tar et
%1 KHHKIK
%ZH 1 1 HH
%1 1 1 1
[6]
\item[e)] Draw the truth table of the unit resulting from part d).
\end{itemize}

%---------------------------------------------------%
\newpage
\subsection{Question 3}
In your study of chapter ll of the guide, you should have met many applications of I
neural networks.
Describe three applications of your choice, at least one of which is of a Hopfield Net. In
each case give:
\begin{itemize}
\item[a)] The aims and objectives ofthe application
[6]
\item[b)] The difficulties faced and how successful it has been, indicating how success is
measured for that application
[6]
\item[c)] The architecture of the neural network used and typical values of the parameters
needed to replicate the application.
[I3]
\end{itemize}
[25 overall]
%---------------------------------------------------%
\subsection{Question 4}
\begin{itemize}
\item[a)] Describe a typical Kohonen-Grossberg network, explaining the different roles played
by each layer.
[5]
\item[b)]  Giving appropriate equations, sketch the algoritlnn used to train the Kohonen layer in
such a network. Define each symbol appearing in the equations.
[8]
\item[c)] The two units of a Kohonen layer are given by (-0.1, 0.2, -0.3, 0.4) and (-0.2, 0.3,
-0.4, 0.5). The network is to be trained using a set of examples; the first one of which is:
(0.3, -0.4, 0.5, -0.6). Showing all of your working, calculate the units resulting from the
presentation of just this one example if the leaming rate is 0.3.
[12]
\end{itemize}

%---------------------------------------------------%

\end{document}

\end{document}
