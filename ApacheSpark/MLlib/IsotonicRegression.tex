Regression - spark.mllib
Isotonic regression
Isotonic regression belongs to the family of regression algorithms. Formally isotonic regression is a problem where given a finite set of real numbers Y=y1,y2,...,ynY=y1,y2,...,yn representing observed responses and X=x1,x2,...,xnX=x1,x2,...,xn the unknown response values to be fitted finding a function that minimises

f(x)=∑i=1nwi(yi−xi)2(1)
(1)f(x)=∑i=1nwi(yi−xi)2

with respect to complete order subject to x1≤x2≤...≤xnx1≤x2≤...≤xn where wiwi are positive weights. The resulting function is called isotonic regression and it is unique. It can be viewed as least squares problem under order restriction. Essentially isotonic regression is a monotonic function best fitting the original data points.

spark.mllib supports a pool adjacent violators algorithm which uses an approach to parallelizing isotonic regression. The training input is a RDD of tuples of three double values that represent label, feature and weight in this order. Additionally IsotonicRegression algorithm has one optional parameter called isotonicisotonic defaulting to true. This argument specifies if the isotonic regression is isotonic (monotonically increasing) or antitonic (monotonically decreasing).

Training returns an IsotonicRegressionModel that can be used to predict labels for both known and unknown features. The result of isotonic regression is treated as piecewise linear function. The rules for prediction therefore are:

If the prediction input exactly matches a training feature then associated prediction is returned. In case there are multiple predictions with the same feature then one of them is returned. Which one is undefined (same as java.util.Arrays.binarySearch).
If the prediction input is lower or higher than all training features then prediction with lowest or highest feature is returned respectively. In case there are multiple predictions with the same feature then the lowest or highest is returned respectively.
If the prediction input falls between two training features then prediction is treated as piecewise linear function and interpolated value is calculated from the predictions of the two closest features. In case there are multiple values with the same feature then the same rules as in previous point are used.
Examples
Scala
Java
Python
Data are read from a file where each line has a format label,feature i.e. 4710.28,500.00. The data are split to training and testing set. Model is created using the training set and a mean squared error is calculated from the predicted labels and real labels in the test set.

Refer to the IsotonicRegression Scala docs and IsotonicRegressionModel Scala docs for details on the API.

import org.apache.spark.mllib.regression.{IsotonicRegression, IsotonicRegressionModel}

val data = sc.textFile("data/mllib/sample_isotonic_regression_data.txt")

// Create label, feature, weight tuples from input data with weight set to default value 1.0.
val parsedData = data.map { line =>
  val parts = line.split(',').map(_.toDouble)
  (parts(0), parts(1), 1.0)
}

// Split data into training (60%) and test (40%) sets.
val splits = parsedData.randomSplit(Array(0.6, 0.4), seed = 11L)
val training = splits(0)
val test = splits(1)

// Create isotonic regression model from training data.
// Isotonic parameter defaults to true so it is only shown for demonstration
val model = new IsotonicRegression().setIsotonic(true).run(training)

// Create tuples of predicted and real labels.
val predictionAndLabel = test.map { point =>
  val predictedLabel = model.predict(point._2)
  (predictedLabel, point._1)
}

// Calculate mean squared error between predicted and real labels.
val meanSquaredError = predictionAndLabel.map { case (p, l) => math.pow((p - l), 2) }.mean()
println("Mean Squared Error = " + meanSquaredError)

// Save and load model
model.save(sc, "target/tmp/myIsotonicRegressionModel")
val sameModel = IsotonicRegressionModel.load(sc, "target/tmp/myIsotonicRegressionModel")
Find full example code at "examples/src/main/scala/org/apache/spark/examples/mllib/IsotonicRegressionExample.scala" in the Spark repo.
