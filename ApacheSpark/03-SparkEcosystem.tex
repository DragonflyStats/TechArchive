	\documentclass[a4paper,12pt]{article}
%%%%%%%%%%%%%%%%%%%%%%%%%%%%%%%%%%%%%%%%%%%%%%%%%%%%%%%%%%%%%%%%%%%%%%%%%%%%%%%%%%%%%%%%%%%%%%%%%%%%%%%%%%%%%%%%%%%%%%%%%%%%%%%%%%%%%%%%%%%%%%%%%%%%%%%%%%%%%%%%%%%%%%%%%%%%%%%%%%%%%%%%%%%%%%%%%%%%%%%%%%%%%%%%%%%%%%%%%%%%%%%%%%%%%%%%%%%%%%%%%%%%%%%%%%%%
\usepackage{eurosym}
\usepackage{vmargin}
\usepackage{amsmath}
\usepackage{graphics}
\usepackage{epsfig}
\usepackage{subfigure}
\usepackage{enumerate}
\usepackage{fancyhdr}

\setcounter{MaxMatrixCols}{10}
%TCIDATA{OutputFilter=LATEX.DLL}
%TCIDATA{Version=5.00.0.2570}
%TCIDATA{<META NAME="SaveForMode"CONTENT="1">}
%TCIDATA{LastRevised=Wednesday, February 23, 201113:24:34}
%TCIDATA{<META NAME="GraphicsSave" CONTENT="32">}
%TCIDATA{Language=American English}

\pagestyle{fancy}
\setmarginsrb{20mm}{0mm}{20mm}{25mm}{12mm}{11mm}{0mm}{11mm}
\lhead{Apache Spark} \rhead{Kevin O'Brien} \chead{The Spark Ecosystem} %\input{tcilatex}

\begin{document}
\secton*{Spark Ecosystem}

Other than Spark Core API, there are additional libraries that are part of the Spark ecosystem and provide additional capabilities in Big Data analytics and Machine Learning areas.

These libraries include:

\begin{itemize}
\item Spark Streaming:
Spark Streaming can be used for processing the real-time streaming data. This is based on micro batch style of computing and processing. It uses the DStream which is basically a series of RDDs, to process the real-time data.
\item Spark SQL:
Spark SQL provides the capability to expose the Spark datasets over JDBC API and allow running the SQL like queries on Spark data using traditional BI and visualization tools. Spark SQL allows the users to ETL their data from different formats it’s currently in (like JSON, Parquet, a Database), transform it, and expose it for ad-hoc querying.
\item Spark MLlib:
MLlib is Spark’s scalable machine learning library consisting of common learning algorithms and utilities, including classification, regression, clustering, collaborative filtering, dimensionality reduction, as well as underlying optimization primitives.
\item Spark GraphX:
GraphX is the new (alpha) Spark API for graphs and graph-parallel computation. At a high level, GraphX extends the Spark RDD by introducing the Resilient Distributed Property Graph: a directed multi-graph with properties attached to each vertex and edge. To support graph computation, GraphX exposes a set of fundamental operators (e.g., subgraph, joinVertices, and aggregateMessages) as well as an optimized variant of the Pregel API. In addition, GraphX includes a growing collection of graph algorithms and builders to simplify graph analytics tasks.
\end{itemize}
%-------------------------------------------------------------------------------%
Outside of these libraries, there are others like BlinkDB and Tachyon.

\begin{itemize}
\item BlinkDB is an approximate query engine and can be used for running interactive SQL queries on large volumes of data. It allows users to trade-off query accuracy for response time. It works on large data sets by running queries on data samples and presenting results annotated with meaningful error bars.

\item Tachyon is a memory-centric distributed file system enabling reliable file sharing at memory-speed across cluster frameworks, such as Spark and MapReduce. It caches working set files in memory, thereby avoiding going to disk to load datasets that are frequently read. This enables different jobs/queries and frameworks to access cached files at memory speed.
\end{itemize}
And there are also integration adapters with other products like Cassandra (Spark Cassandra Connector) and R (SparkR). With Cassandra Connector, you can use Spark to access data stored in a Cassandra database and perform data analytics on that data.

Following diagram (Figure 1) shows how these different libraries in Spark ecosystem are related to each other.


\end{document}
