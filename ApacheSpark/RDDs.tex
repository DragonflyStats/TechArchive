\section*{Resilient Distributed Datasets}
Resilient Distributed Datasets (RDD) is a fundamental data structure of Spark. It is an immutable distributed collection of objects. Each dataset in RDD is divided into logical partitions, which may be computed on different nodes of the cluster. RDDs can contain any type of Python, Java, or Scala objects, including user-defined classes.

Formally, an RDD is a read-only, partitioned collection of records. RDDs can be created through deterministic operations on either data on stable storage or other RDDs. RDD is a fault-tolerant collection of elements that can be operated on in parallel.

There are two ways to create RDDs − parallelizing an existing collection in your driver program, or referencing a dataset in an external storage system, such as a shared file system, HDFS, HBase, or any data source offering a Hadoop Input Format.

Spark makes use of the concept of RDD to achieve faster and efficient MapReduce operations. Let us first discuss how MapReduce operations take place and why they are not so efficient.

\end{document}
