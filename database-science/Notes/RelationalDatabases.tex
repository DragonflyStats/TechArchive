

\chapter{Databases}
\subsection{Relational Database}
A relational database is a collection of data items organized as a set of formally-described tables from which data can be accessed or reassembled in many different ways without having to reorganize the database tables. The relational database was invented by E. F. Codd at IBM in 1970.

The standard user and application program interface to a relational database is the structured query language (SQL). SQL statements are used both for interactive queries for information from a relational database and for gathering data for reports.

In addition to being relatively easy to create and access, a relational database has the important advantage of being easy to extend. After the original database creation, a new data category can be added without requiring that all existing applications be modified.

A relational database is a set of tables containing data fitted into predefined categories. Each table (which is sometimes called a relation) contains one or more data categories in columns. Each row contains a unique instance of data for the categories defined by the columns. For example, a typical business order entry database would include a table that described a customer with columns for name, address, phone number, and so forth. Another table would describe an order: product, customer, date, sales price, and so forth. A user of the database could obtain a view of the database that fitted the user's needs. For example, a branch office manager might like a view or report on all customers that had bought products after a certain date. A financial services manager in the same company could, from the same tables, obtain a report on accounts that needed to be paid.

When creating a relational database, you can define the domain of possible values in a data column and further constraints that may apply to that data value. For example, a domain of possible customers could allow up to ten possible customer names but be constrained in one table to allowing only three of these customer names to be specifiable.

The definition of a relational database results in a table of metadata or formal descriptions of the tables, columns, domains, and constraints.


\subsection{denormalization}

In a relational database, denormalization is an approach to speeding up read performance (data retrieval) in which the administrator selectively adds back specific instances of redundant data after the data structure has been normalized. A denormalized database should not be confused with a database that has never been normalized.

During normalization, the database designer stores different but related types of data in separate logical tables called relations. When a query combines data from multiple tables into a single result table, it is called a join. Multiple joins in the same query can have a negative impact on performance. Introducing denormalization and adding back a small number of redundancies can be a useful for cutting down on the number of joins.

After data has been duplicated, the database designer must take into account how multiple instances of the data will be maintained. One way to denormalize a database is to allow the database management system (DBMS) to store redundant information on disk. This has the added benefit of ensuring the consistency of redundant copies. Another approach is to denormalize the actual logical data design, but this can quickly lead to inconsistent data. 

Rules called constraints can be used to specify how redundant copies of information are synchronized, but they increase the complexity of the database design and also run the risk of impacting write performance.


