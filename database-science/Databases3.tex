\documentclass[12pts]{article}
\usepackage{amsmath}
\usepackage{framed}

%opening
\title{Databases}
\author{Kevin O'Brien}

\begin{document}
\maketitle
\subsection*{Part 1 [6]}
A database is defined as “a shared collection of logically related persistent data as part of the
information system of an organisation”.\\ Explain in brief the meaning of “\textbf{shared}”, “\textbf{logically
related}” and “\textbf{persistent}” in this definition.
\subsection*{Part 2 [6]}
Define the notion of “candidate key”. Can a relation have more than one candidate key? Give
an example. Define the notion of “\textbf{primary key}” in terms of the “\textbf{candidate key}”.
\subsection*{Part 3 [3]}
Explain in brief what is it meant by \textit{\textbf{program data independence}} in the context of database
systems.
\subsection*{Part 4 [3]}
Enumerate three benefits of the database approach to data management (as opposed to a
bare file based approach).

\end{document}
