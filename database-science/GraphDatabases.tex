\subsection{Graph Database}
A graph database is a database that uses graph structures with nodes, edges, and properties to represent and store data. By definition, a graph database is any storage system that provides index-free adjacency. This means that every element contains a direct pointer to its adjacent element and no index lookups are necessary. General graph databases that can store any graph are distinct from specialized graph databases such as triplestores and network databases.

Compared with relational databases, graph databases are often faster for associative data sets, and map more directly to the structure of object-oriented applications. They can scale more naturally to large data sets as they do not typically require expensive join operations. As they depend less on a rigid schema, they are more suitable to manage ad-hoc and changing data with evolving schemas. Conversely, relational databases are typically faster at performing the same operation on large numbers of data elements.

Graph databases are a powerful tool for graph-like queries, for example computing the shortest path between two nodes in the graph. Other graph-like queries can be performed over a graph database in a natural way (for example graph's diameter computations or community detection).

%---------------------------------------------------------------------------- %
