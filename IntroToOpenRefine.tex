OpenRefine, formerly called Google Refine, is a standalone open source desktop application for data cleanup and transformation to other formats, the activity known as data wrangling.[3] It is similar to spreadsheet applications (and can work with spreadsheet file formats); however, it behaves more like a database.

It operates on rows of data which have cells under columns, which is very similar to relational database tables. One OpenRefine project is one table. The user can filter the rows to display using facets that define filtering criteria (for example, showing rows where a given column is not empty). Unlike spreadsheets, most operations in OpenRefine are done on all visible rows: transformation of all cells in all rows under one column,[4] creation of a new column based on existing column data, etc. All actions that were done on a dataset are stored in a project and can be replayed on another dataset.

Unlike spreadsheets, no formulas are stored in the cells, but formulas are used to transform the data, and transformation is done only once.[5] Transformation expressions can be written in Google Refine Expression Language (GREL),[6] Jython (i.e. Python) and Clojure.[7]

The program has a web user interface. However, it is not hosted on the web (SAAS), but is available for download and use on the local machine. When starting OpenRefine, it starts a web server and starts a browser to open the web UI powered by this web server.
