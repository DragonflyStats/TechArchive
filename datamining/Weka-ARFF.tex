Attribute-Relation File Format (ARFF)

November 1st, 2008

This documentation is superceded by the Wiki article ARFF.
April 1st, 2002

An ARFF (Attribute-Relation File Format) file is an ASCII text file that describes a list of instances sharing a set of attributes. ARFF files were developed by the Machine Learning Project at the Department of Computer Science of The University of Waikato for use with the Weka machine learning software. This document descibes the version of ARFF used with Weka versions 3.2 to 3.3; this is an extension of the ARFF format as described in the data mining book written by Ian H. Witten and Eibe Frank (the new additions are string attributes, date attributes, and sparse instances).

This explanation was cobbled together by Gordon Paynter (gordon.paynter at ucr.edu) from the Weka 2.1 ARFF description, email from Len Trigg (lenbok at myrealbox.com) and Eibe Frank (eibe at cs.waikato.ac.nz), and some datasets. It has been edited by Richard Kirkby (rkirkby at cs.waikato.ac.nz). Contact Len if you're interested in seeing the ARFF 3 proposal.

Overview

ARFF files have two distinct sections. The first section is the Header information, which is followed the Data information.

The Header of the ARFF file contains the name of the relation, a list of the attributes (the columns in the data), and their types. An example header on the standard IRIS dataset looks like this:

   % 1. Title: Iris Plants Database
   % 
   % 2. Sources:
   %      (a) Creator: R.A. Fisher
   %      (b) Donor: Michael Marshall (MARSHALL%PLU@io.arc.nasa.gov)
   %      (c) Date: July, 1988
   % 
   @RELATION iris

   @ATTRIBUTE sepallength  NUMERIC
   @ATTRIBUTE sepalwidth   NUMERIC
   @ATTRIBUTE petallength  NUMERIC
   @ATTRIBUTE petalwidth   NUMERIC
   @ATTRIBUTE class        {Iris-setosa,Iris-versicolor,Iris-virginica}
  
The Data of the ARFF file looks like the following:

   @DATA
   5.1,3.5,1.4,0.2,Iris-setosa
   4.9,3.0,1.4,0.2,Iris-setosa
   4.7,3.2,1.3,0.2,Iris-setosa
   4.6,3.1,1.5,0.2,Iris-setosa
   5.0,3.6,1.4,0.2,Iris-setosa
   5.4,3.9,1.7,0.4,Iris-setosa
   4.6,3.4,1.4,0.3,Iris-setosa
   5.0,3.4,1.5,0.2,Iris-setosa
   4.4,2.9,1.4,0.2,Iris-setosa
   4.9,3.1,1.5,0.1,Iris-setosa
  
Lines that begin with a % are comments. The @RELATION, @ATTRIBUTE and @DATA declarations are case insensitive.

Examples

Several well-known machine learning datasets are distributed with Weka in the $WEKAHOME/data directory as ARFF files.

The ARFF Header Section

The ARFF Header section of the file contains the relation declaration and attribute declarations.

The @relation Declaration

The relation name is defined as the first line in the ARFF file. The format is:

    @relation <relation-name>
   
where <relation-name> is a string. The string must be quoted if the name includes spaces.
The @attribute Declarations

Attribute declarations take the form of an orderd sequence of @attribute statements. Each attribute in the data set has its own @attribute statement which uniquely defines the name of that attribute and it's data type. The order the attributes are declared indicates the column position in the data section of the file. For example, if an attribute is the third one declared then Weka expects that all that attributes values will be found in the third comma delimited column.

The format for the @attribute statement is:

    @attribute <attribute-name> <datatype>
   
where the <attribute-name> must start with an alphabetic character. If spaces are to be included in the name then the entire name must be quoted.
The <datatype> can be any of the four types currently (version 3.2.1) supported by Weka:

numeric
<nominal-specification>
string
date [<date-format>]
where <nominal-specification> and <date-format> are defined below. The keywords numeric, string and date are case insensitive.
Numeric attributes

Numeric attributes can be real or integer numbers.

Nominal attributes

Nominal values are defined by providing an <nominal-specification> listing the possible values: {<nominal-name1>, <nominal-name2>, <nominal-name3>, ...}

For example, the class value of the Iris dataset can be defined as follows:

    @ATTRIBUTE class        {Iris-setosa,Iris-versicolor,Iris-virginica}
   
Values that contain spaces must be quoted.

String attributes

String attributes allow us to create attributes containing arbitrary textual values. This is very useful in text-mining applications, as we can create datasets with string attributes, then write Weka Filters to manipulate strings (like StringToWordVectorFilter). String attributes are declared as follows:

    @ATTRIBUTE LCC    string
   
Date attributes

Date attribute declarations take the form:

    @attribute <name> date [<date-format>]
   
where <name> is the name for the attribute and <date-format> is an optional string specifying how date values should be parsed and printed (this is the same format used by SimpleDateFormat). The default format string accepts the ISO-8601 combined date and time format: "yyyy-MM-dd'T'HH:mm:ss".
Dates must be specified in the data section as the corresponding string representations of the date/time (see example below).

ARFF Data Section

The ARFF Data section of the file contains the data declaration line and the actual instance lines.

The @data Declaration

The @data declaration is a single line denoting the start of the data segment in the file. The format is:

    @data
   
The instance data

Each instance is represented on a single line, with carriage returns denoting the end of the instance.

Attribute values for each instance are delimited by commas. They must appear in the order that they were declared in the header section (i.e. the data corresponding to the nth @attribute declaration is always the nth field of the attribute).

Missing values are represented by a single question mark, as in:

    @data
    4.4,?,1.5,?,Iris-setosa
   
Values of string and nominal attributes are case sensitive, and any that contain space must be quoted, as follows:

    @relation LCCvsLCSH

    @attribute LCC string
    @attribute LCSH string

    @data
    AG5,   'Encyclopedias and dictionaries.;Twentieth century.'
    AS262, 'Science -- Soviet Union -- History.'
    AE5,   'Encyclopedias and dictionaries.'
    AS281, 'Astronomy, Assyro-Babylonian.;Moon -- Phases.'
    AS281, 'Astronomy, Assyro-Babylonian.;Moon -- Tables.'
   
Dates must be specified in the data section using the string representation specified in the attribute declaration. For example:

    @RELATION Timestamps

    @ATTRIBUTE timestamp DATE "yyyy-MM-dd HH:mm:ss" 

    @DATA 
    "2001-04-03 12:12:12"
    "2001-05-03 12:59:55"
   
Sparse ARFF files

Sparse ARFF files are very similar to ARFF files, but data with value 0 are not be explicitly represented.

Sparse ARFF files have the same header (i.e @relation and @attribute tags) but the data section is different. Instead of representing each value in order, like this:

    @data
    0, X, 0, Y, "class A"
    0, 0, W, 0, "class B"
   
the non-zero attributes are explicitly identified by attribute number and their value stated, like this:
    @data
    {1 X, 3 Y, 4 "class A"}
    {2 W, 4 "class B"}
   
Each instance is surrounded by curly braces, and the format for each entry is: <index> <space> <value> where index is the attribute index (starting from 0).

Note that the omitted values in a sparse instance are 0, they are not "missing" values! If a value is unknown, you must explicitly represent it with a question mark (?).

Warning: There is a known problem saving SparseInstance objects from datasets that have string attributes. In Weka, string and nominal data values are stored as numbers; these numbers act as indexes into an array of possible attribute values (this is very efficient). However, the first string value is assigned index 0: this means that, internally, this value is stored as a 0. When a SparseInstance is written, string instances with internal value 0 are not output, so their string value is lost (and when the arff file is read again, the default value 0 is the index of a different string value, so the attribute value appears to change). To get around this problem, add a dummy string value at index 0 that is never used whenever you declare string attributes that are likely to be used in SparseInstance objects and saved as Sparse ARFF files.
