
\documentclass[12pt]{article}

%opening
\title{Data Science}
\author{Kevin O'Brien}

\begin{document}
	
\section*{Knowledge Management}

\begin{itemize}
\item Knowledge management software is a tool, or set of tools, used to electronically collect, organize, share, and review business information among a variety of users. The concept of knowledge management (KM) developed in the 1990s as businesses struggled with effective ways to manage and retrieve business information. Enterprise content management systems were developed, along with specific knowledge management software to facilitate the collection and distribution of knowledge. Currently, there are hundreds of different KM software tools, each designed for a specific type of business.


\item Diverse groups such as customer service, information technology (IT), retail, science, engineering, government, and education may need specialized knowledge management software. For example, some KM software manages information about clients or customers, to improve sales and customer support. Another type can gather information about employee skills, talents, education, and business training, to tap into expertise that may have gone unnoticed in the past. Some tools manage work flow, provide training, or standardize processes through step-by-step instructions or manuals.


\item Knowledge management begins with the process of gathering available information. Paper documents, emails, .pdf or word processing files, drawings or graphics, web content, handwritten documents, microfilm or microfiche, and digital files are all possible inputs. A subject matter expert's knowledge can also be written down and added to the knowledge base. Special tools can convert various information inputs into a standard format and improve the quality of the resulting documentation. The resulting knowledge set is usually stored in a powerful database.


\item Once information has been gathered, other knowledge management software tools are used to manage the information. Documents and files can be linked, indexed, and categorized so they are easily scanned and accessed by retrieval tools. Some information may need version control so users are sure to use only the most up-to-date information, while previous versions are archived. KM tools allow multiple users to collaborate on creating information and managing a review and approval process. Other features may include access controls, archive and retrieval capabilities, and conversion to various output types.


\item There are many business benefits from using knowledge management software. It reduces the need for companies to handle and store large amounts of paper or records in various formats with various storage requirements. Information can be retrieved faster, from a wider variety of sources, and can be shared and controlled more easily. KM tools allow users to gather information about the information, such as metrics, productivity, recurring or common problems, and work flow breakdowns. For these reasons, knowledge management software continues to grow in popularity, complexity, and practical application.
\end{itemize}



\end{document}
