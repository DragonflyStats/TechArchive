\documentclass[12pt]{article}

%opening
\title{Data Science}
\author{Kevin O'Brien}

\begin{document}
	
	\section*{Web-mining}
	\begin{itemize}
	
	\item More than ever, entities and individuals alike are using the World Wide Web to conduct a host of business and personal transactions. As a result, companies are increasingly employing Web data mining tools and techniques in order to find ways to improve their bottom lines and grow their customer base.
	\item  Web data mining involves the process of collecting and summarizing data from a Web site’s hyperlink structure, page content, or usage log in order to identify patterns. Using Web data mining, a company can identify a potential competitor, improve customer service, or target customer needs and expectations. A government agency may also seek to uncover terrorist threats or other criminal activities through the use of a Web data mining application.
	
\item	Some common Web data mining techniques include Web content mining, Web usage mining, and Web structure mining. Web content mining examines the subject matter of a Web site. For example, Web content miners may analyze a site's audio, text, images, and video features. Web content miners typically focus on a site’s textual information more than other site features. Natural language processing and information retrieval are two data mining techniques often used by Web content miners.
	
\item Web usage mining is usually an automated process whereby Web servers collect and report user access patterns in server access logs. A company may, for example, use a Web usage data mining tool to report on server access logs and user registration information in order to create a more effective Web site structure. Web structure mining studies the node and connection structure of Web sites. It can be useful in identifying similarities and relationships that exist among different Web sites. Web structure mining often involves uncovering patterns from hyperlinks or pulling out document structures on a Web page.
	
\item	Two general data mining techniques that can be employed by Web data miners are data mining association analysis and data mining regression. Data mining association analysis helps uncover noteworthy relationships buried in large data sets. \textbf{Data mining regression} is a statistical technique whereby mathematical formulas are used to predict future results, such as profit margins, house values, or sales figures.
	
\item	Data mining software vendors offer Web data mining tools that can pull out predictive information from large quantities of data. Businesses often use these software mining tools to analyze specific data sets regarding consumer behavior. Using the results of the data analysis, companies are able to forecast future business trends.
	
	\end{itemize}
\end{document}
