Scala - Files I/O
Advertisements
 Previous Page Next Page  
Scala is open to make use of any Java objects and java.io.File is one of the objects which can be used in Scala programming to read and write files.

The following is an example program to writing to a file.

Example
import java.io._

object Demo {
   def main(args: Array[String]) {
      val writer = new PrintWriter(new File("test.txt" ))

      writer.write("Hello Scala")
      writer.close()
   }
}
Save the above program in Demo.scala. The following commands are used to compile and execute this program.

Command
\>scalac Demo.scala
\>scala Demo
It will create a file named Demo.txt in the current directory, where the program is placed. The following is the content of that file.

Output
Hello Scala
Reading a Line from Command Line
Sometime you need to read user input from the screen and then proceed for some further processing. Following example program shows you how to read input from the command line.

Example
object Demo {
   def main(args: Array[String]) {
      print("Please enter your input : " )
      val line = Console.readLine
      
      println("Thanks, you just typed: " + line)
   }
}
Save the above program in Demo.scala. The following commands are used to compile and execute this program.

Command
\>scalac Demo.scala
\>scala Demo
Output
Please enter your input : Scala is great
Thanks, you just typed: Scala is great
Reading File Content
Reading from files is really simple. You can use Scala's Source class and its companion object to read files. Following is the example which shows you how to read from "Demo.txt" file which we created earlier.

Example
import scala.io.Source

object Demo {
   def main(args: Array[String]) {
      println("Following is the content read:" )

      Source.fromFile("Demo.txt" ).foreach { 
         print 
      }
   }
}
Save the above program in Demo.scala. The following commands are used to compile and execute this program.

Command
\>scalac Demo.scala
\>scala Demo
Output
Following is the content read:
Hello Scala
