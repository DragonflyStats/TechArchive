Scala - Basic Syntax
Advertisements
 Previous Page Next Page  
If you have a good understanding on Java, then it will be very easy for you to learn Scala. The biggest syntactic difference between Scala and Java is that the ';' line end character is optional.

When we consider a Scala program, it can be defined as a collection of objects that communicate via invoking each other’s methods. Let us now briefly look into what do class, object, methods and instant variables mean.

Object − Objects have states and behaviors. An object is an instance of a class. Example − A dog has states - color, name, breed as well as behaviors - wagging, barking, and eating.

Class − A class can be defined as a template/blueprint that describes the behaviors/states that are related to the class.

Methods − A method is basically a behavior. A class can contain many methods. It is in methods where the logics are written, data is manipulated and all the actions are executed.

Fields − Each object has its unique set of instant variables, which are called fields. An object's state is created by the values assigned to these fields.

Closure − A closure is a function, whose return value depends on the value of one or more variables declared outside this function.

Traits − A trait encapsulates method and field definitions, which can then be reused by mixing them into classes. Traits are used to define object types by specifying the signature of the supported methods.

First Scala Program
We can execute a Scala program in two modes: one is interactive mode and another is script mode.

Interactive Mode
Open the command prompt and use the following command to open Scala.

\>scala
If Scala is installed in your system, the following output will be displayed −

Welcome to Scala version 2.9.0.1
Type in expressions to have them evaluated.
Type :help for more information.
Type the following text to the right of the Scala prompt and press the Enter key −

scala> println("Hello, Scala!");
It will produce the following result −

Hello, Scala!
Script Mode
Use the following instructions to write a Scala program in script mode. Open notepad and add the following code into it.

Example
object HelloWorld {
   /* This is my first java program.  
   * This will print 'Hello World' as the output
   */
   def main(args: Array[String]) {
      println("Hello, world!") // prints Hello World
   }
}
Save the file as − HelloWorld.scala.

Open the command prompt window and go to the directory where the program file is saved. The ‘scalac’ command is used to compile the Scala program and it will generate a few class files in the current directory. One of them will be called HelloWorld.class. This is a bytecode which will run on Java Virtual Machine (JVM) using ‘scala’ command.

Use the following command to compile and execute your Scala program.

\> scalac HelloWorld.scala
\> scala HelloWorld
Output
Hello, World!
Basic Syntax
The following are the basic syntaxes and coding conventions in Scala programming.

Case Sensitivity − Scala is case-sensitive, which means identifier Hello and hello would have different meaning in Scala.

Class Names − For all class names, the first letter should be in Upper Case. If several words are used to form a name of the class, each inner word's first letter should be in Upper Case.

Example − class MyFirstScalaClass.

Method Names − All method names should start with a Lower Case letter. If multiple words are used to form the name of the method, then each inner word's first letter should be in Upper Case.

Example − def myMethodName()

Program File Name − Name of the program file should exactly match the object name. When saving the file you should save it using the object name (Remember Scala is case-sensitive) and append ‘.scala’ to the end of the name. (If the file name and the object name do not match your program will not compile).

Example − Assume 'HelloWorld' is the object name. Then the file should be saved as 'HelloWorld.scala'.

def main(args: Array[String]) − Scala program processing starts from the main() method which is a mandatory part of every Scala Program.

Scala Identifiers
All Scala components require names. Names used for objects, classes, variables and methods are called identifiers. A keyword cannot be used as an identifier and identifiers are case-sensitive. Scala supports four types of identifiers.

Alphanumeric Identifiers
An alphanumeric identifier starts with a letter or an underscore, which can be followed by further letters, digits, or underscores. The '$' character is a reserved keyword in Scala and should not be used in identifiers.

Following are legal alphanumeric identifiers −

age, salary, _value,  __1_value
Following are illegal identifiers −

$salary, 123abc, -salary
Operator Identifiers
An operator identifier consists of one or more operator characters. Operator characters are printable ASCII characters such as +, :, ?, ~ or #.

Following are legal operator identifiers −

+ ++ ::: <?> :>
The Scala compiler will internally "mangle" operator identifiers to turn them into legal Java identifiers with embedded $ characters. For instance, the identifier :-> would be represented internally as $colon$minus$greater.

Mixed Identifiers
A mixed identifier consists of an alphanumeric identifier, which is followed by an underscore and an operator identifier.

Following are legal mixed identifiers −

unary_+,  myvar_=
Here, unary_+ used as a method name defines a unary + operator and myvar_= used as method name defines an assignment operator (operator overloading).

Literal Identifiers
A literal identifier is an arbitrary string enclosed in back ticks (` . . . `).

Following are legal literal identifiers −

`x` `<clinit>` `yield`
Scala Keywords
The following list shows the reserved words in Scala. These reserved words may not be used as constant or variable or any other identifier names.

abstract	case	catch	class
def	do	else	extends
false	final	finally	for
forSome	if	implicit	import
lazy	match	new	Null
object	override	package	private
protected	return	sealed	super
this	throw	trait	Try
true	type	val	Var
while	with	yield	 
-	:	=	=>
<-	<:	<%	>:
#	@		
Comments in Scala
Scala supports single-line and multi-line comments very similar to Java. Multi-line comments may be nested, but are required to be properly nested. All characters available inside any comment are ignored by Scala compiler.

object HelloWorld {
   /* This is my first java program.  
    * This will print 'Hello World' as the output
    * This is an example of multi-line comments.
    */
   def main(args: Array[String]) {
      // Prints Hello World
      // This is also an example of single line comment.
      println("Hello, world!") 
   }
}
Blank Lines and Whitespace
A line containing only whitespace, possibly with a comment, is known as a blank line, and Scala totally ignores it. Tokens may be separated by whitespace characters and/or comments.

Newline Characters
Scala is a line-oriented language where statements may be terminated by semicolons (;) or newlines. A semicolon at the end of a statement is usually optional. You can type one if you want but you don't have to if the statement appears by itself on a single line. On the other hand, a semicolon is required if you write multiple statements on a single line. Below syntax is the usage of multiple statements.

val s = "hello"; println(s)
Scala Packages
A package is a named module of code. For example, the Lift utility package is net.liftweb.util. The package declaration is the first non-comment line in the source file as follows −

package com.liftcode.stuff
Scala packages can be imported so that they can be referenced in the current compilation scope. The following statement imports the contents of the scala.xml package −

import scala.xml._
You can import a single class and object, for example, HashMap from the scala.collection.mutable package −

import scala.collection.mutable.HashMap
You can import more than one class or object from a single package, for example, TreeMap and TreeSet from the scala.collection.immutable package −

import scala.collection.immutable.{TreeMap, TreeSet}
Apply Dynamic
A marker trait that enables dynamic invocations. Instances x of this trait allow method invocations x.meth(args) for arbitrary method names meth and argument lists args as well as field accesses x.field for arbitrary field namesfield. This feature is introduced in Scala-2.10.

If a call is not natively supported by x (i.e. if type checking fails), it is rewritten according to the following rules −

foo.method("blah") ~~> foo.applyDynamic("method")("blah")
foo.method(x = "blah") ~~> foo.applyDynamicNamed("method")(("x", "blah"))
foo.method(x = 1, 2) ~~> foo.applyDynamicNamed("method")(("x", 1), ("", 2))
foo.field ~~> foo.selectDynamic("field")
foo.varia = 10 ~~> foo.updateDynamic("varia")(10)
foo.arr(10) = 13 ~~> foo.selectDynamic("arr").update(10, 13)
foo.arr(10) ~~> foo.applyDynamic("arr")(10)

%====================================================%
