CHAID is a type of decision tree technique, based upon adjusted significance testing (Bonferroni testing). The technique was developed in South Africa and was published in 1980 by Gordon V. Kass, who had completed a PhD thesis on this topic. CHAID can be used for prediction (in a similar fashion to regression analysis, this version of CHAID being originally known as XAID) as well as classification, and for detection of interaction between variables. CHAID stands for CHi-squared Automatic Interaction Detection, based upon a formal extension of the US AID (Automatic Interaction Detection) and THAID (THeta Automatic Interaction Detection) procedures of the 1960s and 70s, which in turn were extensions of earlier research, including that performed in the UK in the 1950s.

In practice, CHAID is often used in the context of direct marketing to select groups of consumers and predict how their responses to some variables affect other variables, although other early applications were in the field of medical and psychiatric research.

Like other decision trees, CHAID's advantages are that its output is highly visual and easy to interpret. Because it uses multiway splits by default, it needs rather large sample sizes to work effectively, since with small sample sizes the respondent groups can quickly become too small for reliable analysis.

One important advantage of CHAID over alternatives such as multiple regression is that it is non-parametric.
