 tutorialspoint
 We are hiring
 Search
Scala Tutorial
Scala Tutorial
Scala - Home
Scala - Overview
Scala - Environment Setup
Scala - Basic Syntax
Scala - Data Types
Scala - Variables
Scala - Classes & Objects
Scala - Access Modifiers
Scala - Operators
Scala - IF ELSE
Scala - Loop Statements
Scala - Functions
Scala - Closures
Scala - Strings
Scala - Arrays
Scala - Collections
Scala - Traits
Scala - Pattern Matching
Scala - Regular Expressions
Scala - Exception Handling
Scala - Extractors
Scala - Files I/O
Scala Useful Resources
Scala - Quick Guide
Scala - Useful Resources
Scala - Discussion
Selected Reading
Developer's Best Practices
Questions and Answers
Effective Resume Writing
HR Interview Questions
Computer Glossary
Who is Who
Scala - Data Types
Advertisements
 Previous Page Next Page  
Scala has all the same data types as Java, with the same memory footprint and precision. Following is the table giving details about all the data types available in Scala −

Sr.No	Data Type & Description
1	
Byte

8 bit signed value. Range from -128 to 127

2	
Short

16 bit signed value. Range -32768 to 32767

3	
Int

32 bit signed value. Range -2147483648 to 2147483647

4	
Long

64 bit signed value. -9223372036854775808 to 9223372036854775807

5	
Float

32 bit IEEE 754 single-precision float

6	
Double

64 bit IEEE 754 double-precision float

7	
Char

16 bit unsigned Unicode character. Range from U+0000 to U+FFFF

8	
String

A sequence of Chars

9	
Boolean

Either the literal true or the literal false

10	
Unit

Corresponds to no value

11	
Null

null or empty reference

12	
Nothing

The subtype of every other type; includes no values

13	
Any

The supertype of any type; any object is of type Any

14	
AnyRef

The supertype of any reference type

All the data types listed above are objects. There are no primitive types like in Java. This means that you can call methods on an Int, Long, etc.

Scala Basic Literals
The rules Scala uses for literals are simple and intuitive. This section explains all basic Scala Literals.

Integral Literals
Integer literals are usually of type Int, or of type Long when followed by a L or l suffix. Here are some integer literals −

0
035
21 
0xFFFFFFFF 
0777L
Floating Point Literal
Floating point literals are of type Float when followed by a floating point type suffix F or f, and are of type Double otherwise. Here are some floating point literals −

0.0 
1e30f 
3.14159f 
1.0e100
.1
Boolean Literals
The Boolean literals true and false are members of type Boolean.

Symbol Literals
A symbol literal 'x is a shorthand for the expression scala.Symbol("x"). Symbol is a case class, which is defined as follows.

package scala
final case class Symbol private (name: String) {
   override def toString: String = "'" + name
}
Character Literals
A character literal is a single character enclosed in quotes. The character is either a printable Unicode character or is described by an escape sequence. Here are some character literals −

'a' 
'\u0041'
'\n'
'\t'
String Literals
A string literal is a sequence of characters in double quotes. The characters are either printable Unicode character or are described by escape sequences. Here are some string literals −

"Hello,\nWorld!"
"This string contains a \" character."
Multi-Line Strings
A multi-line string literal is a sequence of characters enclosed in triple quotes """ ... """. The sequence of characters is arbitrary, except that it may contain three or more consecutive quote characters only at the very end.

Characters must not necessarily be printable; newlines or other control characters are also permitted. Here is a multi-line string literal −

"""the present string
spans three
lines."""
Null Values
The null value is of type scala.Null and is thus compatible with every reference type. It denotes a reference value which refers to a special "null" object.

Escape Sequences
The following escape sequences are recognized in character and string literals.

Escape Sequences	Unicode	Description
\b	\u0008	backspace BS
\t	\u0009	horizontal tab HT
\n	\u000c	formfeed FF
\f	\u000c	formfeed FF
\r	\u000d	carriage return CR
\"	\u0022	double quote "
\'	\u0027	single quote .
\\	\u005c	backslash \
A character with Unicode between 0 and 255 may also be represented by an octal escape, i.e., a backslash '\' followed by a sequence of up to three octal characters. Following is the example to show few escape sequence characters −

Example
object Test {
   def main(args: Array[String]) {
      println("Hello\tWorld\n\n" );
   }
} 
When the above code is compiled and executed, it produces the following result −

Output
Hello   World
 Previous Page  Print Next Page  
Advertisements
 img  img  img  img  img  img
 Tutorials Point
Write for us FAQ's Helping Contact
© Copyright 2016. All Rights Reserved.

Enter email for newsletter
  go
