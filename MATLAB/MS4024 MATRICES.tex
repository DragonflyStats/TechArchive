\documentclass[a4paper,12pt]{article}
%%%%%%%%%%%%%%%%%%%%%%%%%%%%%%%%%%%%%%%%%%%%%%%%%%%%%%%%%%%%%%%%%%%%%%%%%%%%%%%%%%%%%%%%%%%%%%%%%%%%%%%%%%%%%%%%%%%%%%%%%%%%%%%%%%%%%%%%%%%%%%%%%%%%%%%%%%%%%%%%%%%%%%%%%%%%%%%%%%%%%%%%%%%%%%%%%%%%%%%%%%%%%%%%%%%%%%%%%%%%%%%%%%%%%%%%%%%%%%%%%%%%%%%%%%%%
\usepackage{eurosym}
\usepackage{vmargin}
\usepackage{amsmath}
\usepackage{graphics}
\usepackage{epsfig}
\usepackage{framed}
\usepackage{subfigure}
\usepackage{fancyhdr}

\setcounter{MaxMatrixCols}{10}
%TCIDATA{OutputFilter=LATEX.DLL}
%TCIDATA{Version=5.00.0.2570}
%TCIDATA{<META NAME="SaveForMode"CONTENT="1">}
%TCIDATA{LastRevised=Wednesday, February 23, 201113:24:34}
%TCIDATA{<META NAME="GraphicsSave" CONTENT="32">}
%TCIDATA{Language=American English}

\pagestyle{fancy}
\setmarginsrb{20mm}{0mm}{20mm}{25mm}{12mm}{11mm}{0mm}{11mm}
\lhead{MS4024 - MATLAB} \rhead{Kevin O'Brien} \chead{Week 8} %\input{tcilatex}

%http://www.electronics.dit.ie/staff/ysemenova/Opto2/CO_IntroLab.pdf
\begin{document}

\tableofcontents
\newpage


\textbf{Remarks:}
\begin{itemize}
\item Write your answer to 2 decimal places, unless otherwise instructed.
\item Some solutions will contain complex components.
\end{itemize}


\section{Creating Matrices}
Recall how to compute matrices, using MATLAB.

%http://www.mathworks.co.uk/products/matlab/examples.html;jsessionid=925253a9b3e21c880ba1fd97e057?file=/products/demos/shipping/matlab/intro.html
%--------------------------------- %
\begin{framed}
\begin{verbatim}
X = [1 2 3 4 6 4 8 4 5]
X =

     1     2     3     4     6     4     8     4     5

Y = [1,3,6;2,7,8;0,3,9]
Y =
  1    3    6
  2    7    8
  0    3    9

\end{verbatim}
\end{framed}

\subsection{Basic Matrix Operations}

For the matrix Y described in the above code, answer the following questions.
\begin{description}
\item[\texttt{trace(Y)}] Compute the \textbf{\textit{trace}} of matrix Y
\item[\texttt{rank(Y)}] Compute the \textbf{\textit{rank}} of matrix Y
\item[\texttt{inv(Y)}] Computes the inverse of matrix Y
\item[\texttt{det(Y)}] Computes the determinant of matrix Y
\item[\texttt{$Y * Y$}] Computes $Y^2$.
\item[\texttt{$Y*Y*Y$}] Computes $Y^3$.
\end{description}

\section{Determining Eigen-values}
In the first instance the command \texttt{\textbf{eig()}} is used to compute the eigen-values of a matrix.The \texttt{\textbf{poly}} function generates a vector containing the coefficients of the characteristic polynomial. Recall that the characteristic polynomial of a matrix A is defined as:

\[det( \lambda I - A )\]

\begin{itemize}
\item[i)] determine the characteristic polynomial of matrix Z. (The full equation, not just the coefficients)
\item[ii)] compute W, the inverse of Z.
\item[iii)] compute the eigenvalues of W.(four decimal places)
\end{itemize}
\begin{framed}
\begin{verbatim}
Z = [1 2 0; 2 5 -1; 4 10 -1];

\end{verbatim}
\end{framed}
\newpage

%--------------------------------- %

\section{Eigen-decomposition}
The command \texttt{\textbf{eig()}} can be used to perform the command \textit{\textbf{eigen-decomposition}}, when used in the manner described below. The command \texttt{[V,D]=eig(A)} produces the V matrix, whose columns are eigenvectors, and the diagonal matrix D whose values are eigenvalues of the matrix A.

\subsection{Example}

\begin{framed}
\begin{verbatim}
A = [5 3 2; 1 4 6; 9 7 2];
[V,D]=eig(A);
\end{verbatim}
\end{framed}
\begin{verbatim}
V =
  0.7217   0.1918   0.7680
  0.5557 - 0.7773 - 0.6388
  0.4127   0.5992   0.0459

D =

Diagonal Matrix

 -3.2847   0        0
  0        1.7486   0
  0        0       12.5361
\end{verbatim}

\begin{itemize}
\item[i)] Compute the characteristic equation of the following matrix.
\item[ii)] Compute the eigenvalues of the matrix.
\item[iii)] Compute the eigenvectors of the matrix.
\end{itemize}
\[\left(
\begin{array}{ccc}
5 & 9 & 13 \\
11 & 7 & 1 \\
8 & 9 & 3 \\
\end{array}
\right)\]
%--------------------------------- %

\section{Solving linear systems of equations}

The MATLAB command sequence \texttt{A$\backslash$b} is used to solve the equation  \textbf{\textit{Ax=b}}.

%\[
%\begin{alignat}
%3x &&\; + \;&& 2y             &&\; - \;&& z  &&\; = \;&& 1 & \\
%2x &&\; - \;&& 2y             &&\; + \;&& 4z &&\; = \;&& -2 & \\
%-x &&\; + \;&& \tfrac{1}{2} y &&\; - \;&& z  &&\; = \;&& 0 &
%\end{alignat}
%\]

Solve the following system of linear equations,using your own names for the matrices.
\[\left(
\begin{array}{ccc}
5 & 9 & 13 \\
11 & 7 & 1 \\
8 & 9 & 3 \\
\end{array}
\right)
\left(
\begin{array}{c}
x_1 \\
x_2 \\
x_3 \\
\end{array}
\right)
=\left(
\begin{array}{c}
5  \\
7  \\
9  \\
\end{array}
\right)\]
%-----------------------%


%-----------------------%
\section{Using Commands}

Given the array A = [2 7 9 7 ; 3 1 5 6 ; 8 1 2 5], explain the results of the following commands:

\begin{framed}
\begin{verbatim}
A = [2 7 9 7 ; 3 1 5 6 ; 8 1 2 5];

\end{verbatim}
\end{framed}

\begin{itemize}
\item[i)] \texttt{A'}
\item[ii)] \texttt{A(:,[1 4])}
\item[iii)] \texttt{A([1 2],[1 2])}
\item[iv)] \texttt{A([2 3],[1 2])}
\item[v)] \texttt{A([2 3],[1 3])}
\item[vi)] \texttt{A([2 3],[3 1])}
\item[vii)] \texttt{reshape(A,2,6)}
\item[viii)] \texttt{size(A)}
\item[ix)] \texttt{flipud(A)}
\item[x)] \texttt{fliplr(A)}
\item[xi)] \texttt{A(end,:)}
\item[xii)]\texttt{[A ; A(end,:)]}
\item[xiii)] \texttt{A(1:2,:)}
\item[xiv)] \texttt{A'(1:3,:)}
\item[xv)]\texttt{[A ; A(1:2,:)]}
\item[xvi)] \texttt{sum(A)}  
\item[xvii)] \texttt{sum(A')}
\item[xviii)] \texttt{sum(A,1)} and \texttt{sum(A,2)}
\item[xiv)] \texttt{[ [ A ; sum(A) ] [ sum(A,2) ; sum(A(:)) ] ]}
\item[xv)] \texttt{A(:)}
\end{itemize}


%----------------------------------------------------- %


\section{More Exercises}
\subsection*{Question 1}
Consider the matrices A and B, given as:
\[A= \left(
\begin{array}{c}
2.4  \\
1.4  \\
1.2  \\
\end{array}
\right)\]

\[B= \left(
\begin{array}{ccc}
2.2  & 1.3 & 1.2  \\
\end{array}
\right)\]

Determine the following matrices.
\begin{itemize}
\item[i.] $C = A \times B$
\item[ii.] $D = B \times A$
\item[iii.] $E = C^{-1}$
\end{itemize}

\subsection*{Question 2}
\begin{itemize}
\item[i.] Determine the rank of the following matrix F.
\end{itemize}
\[\mathbf{F} = \begin{bmatrix}
9 & 13 & 5 \\
1 & 11 & 7 \\
3 & 7 & 2 \\
6 & 0 & 7 \end{bmatrix}\]

\subsection*{Question 3}
compute the determinant of matrix G.
\[G = \left(
\begin{array}{cccc}
1 & 5 & 9 & 13 \\
2 & 11 & 7 & 1 \\
4 & 8 & 10 & 3 \\
6 & 15 & 9 & 8 \\
\end{array}
\right)\]
%\subsection*{Question 4}

C= [1 3 9; 6 7 2; 8 -1 -2];
\end{document}
