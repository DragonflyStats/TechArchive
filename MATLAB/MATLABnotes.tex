This section discusses input and output concepts related to MATLAB.

\section{Some useful output commands}

\begin{description}
\item[\texttt{disp}]
\item[\texttt{fprintf}]
\end{description}
The commands \texttt{load} and \textt{save} which read data from a file into a
a matrix, and write a matrix to file.

The load command works only if there are the same number of values in each line and the
values are of the same type, so that the data can be stored in a matrix. The save command only writes a matrix to a file.

\section{Appending to a file}
%-----------------------------------%

%8.1.1
\section{Opening and closing a file}

Permission Strings
\begin{description}
\item[r]
\item[w]
\item[a]
\end{description}

%------------------------------------%


%8.1.2 
\section{Reading from files}
%------------------------------------%
The \texttt{fgetl} function

%------------------------------------%
The \texttt{fopen} function opens the file for reading

The \texttt{fprintf} function  reads from each line


%------------------------------------%

%8.1.3 Writing to files

the \texttt{fprintf} function actually returns the number of bytes written to a file, so if you do not have to see that number, suppress the output with a semi-colon.


\begin{framed}
\begin{verbatim}
load randmat.dat
\end{verbatim}
\end{framed}
%------------------------------------%
%8.1.4
\section{Appending to Files}
The fprintf function can also be used to append to an existing file.
The permission string is `a`. For example
\begin{framed}
\begin{verbatim}
fid= fopen(`filename`,`a`);
\end{verbatim}
\end{framed}
%------------------------------------%
\textbf{Exercise}\\
create a $3 \times 5$ matrix of random integers, each in the range 1 to 100.
Write this to a file called `myrandmat.dat` in a $3 \times 5$ format using \texttt{fprintf}
, so that the file appears identical to the original matrix.
load the file to confirm that it was created correctly.

%------------------------------------%
% 8.2 Writing and Reading Spreadsheet files

\begin{framed}
\begin{verbatim}
pwd
ranmat=randint(5,3,[1 100])
xlswrite('ranexcel',ranmat)

\end{verbatim}
\end{framed}
The xlsread function will read a spreadsheet file. For example, to read
from the file just created
\begin{framed}
\begin{verbatim}
ssnums =xls('ranexcel')
\end{verbatim}
\end{framed}
In both cases the .xls extension on the filename is the default setting, so there is no need to specify it.


%------------------------------------%
% 8.3 using MAT files for variables

In addition to the data file types, matlab has functions that allow reading
and saving variables from files. These files are called MAT files (because 
the extension on the filename is .mat) and they store the names and contents
of variables. Variables can be written to MAT files, appended to them, and
read from them.

Remark ; MAT files are very different .
%----------%
% 8.3.1 Writing variables to MAT files


%PAge 267
\begin{framed}
\begin{verbatim}

mymat=rand(3,5)
x=1:6;
y=x.^2; 

who

Your Variables are

\end{verbatim}
\end{framed}

% 8.3.3 Reading from a MAT file

%------------------------------------------%
