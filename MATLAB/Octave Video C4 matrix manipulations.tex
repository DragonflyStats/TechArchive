\documentclass{beamer}

\usepackage{amsmath}
\usepackage{amssymb}
\usepackage{graphics}

\begin{document}
%-------------------------------------------------- %
\begin{frame}
\bigskip
{
\Huge
\[ \mbox{Octave}  \]
}
{
\huge
\[ \mbox{Matrices Manipulations (Exercise C.6)}  \]
}

{
\LARGE
\[ \mbox{www.stats-lab.com}  \]
\[ \mbox{Twitter: @StatsLabDublin} \]
}

\end{frame}
%-------------------------------------------------- %
\begin{frame}
\frametitle{Octave: Matrices Manipulations}
{
\Large
Suppose we have the $4 \times 4$ matrix C.
}
{\Large
\[C = \left( \begin{array}{cccc}
2& 7& -5& 5\\
4&-2 & 7& -1\\
6&-3 & 0& 3 \\
4&-2 & 7& 1
\end{array} \right)
\]
}
{
\Large
\begin{enumerate}
\item Replace all negative values of matrix C with 0. (Call this matrix $C1$)
\item Replace all negative values of matrix C with the absolute value of those values. \\(Call this matrix $C2$.)
\end{enumerate}
}
\end{frame}
%-------------------------------------------------- %
\begin{frame}
\frametitle{Octave: Matrices Manipulations}
\LARGE
\textbf{Exercise 1}: Replace all negative values of matrix C with 0. \\

\begin{itemize}
\item The \texttt{max()} command, comparing each value to 0.
\end{itemize} 

\end{frame}
%-------------------------------------------------- %
\begin{frame}
\frametitle{Octave: Matrices Manipulations}
\LARGE
\textbf{Exercise 2}: Replace all negative values of matrix C with the absolute value of those values.\\

\begin{itemize}
\item The \texttt{max()} command, comparing each value to the negation of that value.
\end{itemize} 

\end{frame}
%-------------------------------------------------- 

%-------------------------------------------------- %

\end{document}
