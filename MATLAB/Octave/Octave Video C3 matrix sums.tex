\documentclass{beamer}

\usepackage{amsmath}
\usepackage{amssymb}
\usepackage{graphics}

\begin{document}
%-------------------------------------------------- %
\begin{frame}
\bigskip
{
\Huge
\[ \mbox{Octave}  \]
}
{
\huge
\[ \mbox{Matrix Summations (Exercise C.3)}  \]
}

{
\LARGE
\[ \mbox{www.stats-lab.com}  \]
\[ \mbox{Twitter: @StatsLabDublin} \]
}

\end{frame}
%-------------------------------------------------- %
\begin{frame}
\frametitle{Octave: Summations with Matrices}
{
\Large
Suppose we have the $3 \times 4$ matrix A.
}
{\Large
\[ 
A = \left(
\begin{array}{cccc}
 5 & 2 & 1 & -1 \\ 
 1 & 3 & 2 & 5 \\ 
 -1 & 4 & 7 & 1 \\ 
\end{array} \right)
\]
}
{
\Large
\begin{enumerate}
\item Compute the sum totals for each column,
\item Compute the sum totals for each row,
\item Compute the overall sum total.
\end{enumerate}
}
\end{frame}
%-------------------------------------------------- %
\begin{frame}
\frametitle{Octave: Summations with Matrices}
\LARGE
\textbf{Exercise 1}: Compute the sum totals for each column.\\

\begin{itemize}
\item Default setting of the command \texttt{sum()}, when applied to a matrix.
\end{itemize} 

\end{frame}
%-------------------------------------------------- %
\begin{frame}
\frametitle{Octave: Summations with Matrices}
\LARGE
\textbf{Exercise 2}: Compute the sum totals for each row.\\

\begin{itemize}
\item To work on a row-wise basis, we simply have to make the additional specification ``2" to the \texttt{sum()} command.
\item Alternatively we could use the tranpose operator (i.e.  \texttt{A'})
\end{itemize} 

\end{frame}
%-------------------------------------------------- %
\begin{frame}
\frametitle{Octave: Summations with Matrices}
\LARGE
\textbf{Exercise 3}: Compute the overall sum totals for matrix A.\\

\begin{itemize}
\item To find the sum total of all elements in matrix A, we simply find the sum of one of our previous results.
\item The structure of the command will look like \texttt{sum(sum(...))}.
\end{itemize} 

\end{frame}
%-------------------------------------------------- %

\end{document}
