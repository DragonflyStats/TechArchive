\documentclass[12pt]{article}
\usepackage{framed}
\usepackage{amsmath}
\usepackage{graphics}
%opening
\title{Octave Work Sheet}
\author{StatsLabDublin  (www.stats-lab.com)}

\begin{document}

\maketitle

\section*{Part A : Working with Octave}
\subsection*{Question A.1}
\begin{enumerate}
\item Clear the screen, and change the prompt to the following:
\begin{framed}
\begin{verbatim}
>>:
>>:
\end{verbatim}
\end{framed}
\item Change the \textbf{\textit{Working Directory}} to \textbf{\texttt{C:/WorkArea/Octave}}.
\item Close the current octave session.
\end{enumerate}
%--------------------------------------------%
\newpage

\section*{Part B : Basic Mathematical Operations}
There will be a separate section for Matrices and Linear Algebra.

\subsection*{Question B.1}
\begin{enumerate}
\item Compute the sum of the first 100 positive integers.
\[ \sum^{100}_{i=1} i \]
\item Compute the sum of the squares of the \textbf{first} 100 positive integers.
\[ \sum^{100}_{i=1} i^2 \]
\end{enumerate}
\subsection*{Question B.2}
\begin{enumerate}
\item Determine the prime factors of the following numbers
\begin{itemize}
\item \textbf{8475495}
\item \textbf{934234}
\item \textbf{783932234}
\end{itemize}
\end{enumerate}
\subsection*{Question B.3}  %Complex Numbers
\begin{enumerate}
\item Evaluate the following expression:
{\Large \[ e^{i \pi} \]}
\item Evaluate the following expression:
{\Large \[ \lceil e^{1.5} \rceil \]}
\item Evaluate the following expression:
\[ {\LARGE \lfloor  tan \left( \frac{\pi}{3} \right) \rfloor  }\]
\end{enumerate}
\subsection*{Question B.4}
\begin{enumerate}
\item How many prime numbers are less than 100?
\item Compute the sum of all prime numbers less than 100.
\item Compute the sum of all prime numbers between 50 and 100.
\item Compute the sum of the first 100 prime numbers.
\end{enumerate}

\subsection*{Question B.5}
Let \textbf{P} be the set of all prime numbers less than 10000.
\begin{enumerate}
\item How many numbers are there in set \textbf{P}?
\item For each value of \textbf{P}, determine the set \textbf{P1}, which are the floor function values of the square roots of each value of \textbf{P}. Compute the sum of the values in \textbf{P1}.
\item How many \textbf{\textit{unique}} values are there in \textbf{P1}?
\end{enumerate}

\subsection*{Question B.6}
Suppose the data set $X$ is a randomly selected sample.
\begin{framed}
\begin{verbatim}
X=[ 5.1, 4.9, 4.7, 4.6, 5, 5.4, 4.6, 5, 4.4, 4.9, 5.4,4.8]
 \end{verbatim}
\end{framed}
\begin{enumerate}
\item How many values are there in $X$?
\item Compute the sum of the values in $X$.
\item Calculate the mean, median and mode of $X$.
\item Calculate the standard deviation and variance of $X$.
\item Calculate the maximum value, minimum value and range of $X$.
\end{enumerate}

\subsection*{Question B.7}
Determine the roots of the following polynomials. (Some will have complex roots)
\[ x^3 + 2x^2 - 5x + 4 \]
\[ x^4 - 4x^4 + 6x^2 - 4x + 1 \]
\[ x^5 + 4x^3 + 2x + 1  \]
%------------------------------------------------%
\newpage
\section*{Part C: Matrices and Linear Algebra}

Construct the following Matrices
{
\LARGE
\[ A =  \left( \begin{array}{ccc}
4 & 7 & 5 \\ 
1 & 2 & 5 \\ 
6 & 1 & 3
\end{array}   \right)
\qquad B = \left( \begin{array}{cc}
3 & 5 \\ 
1 & 5 \\ 
5 & 2
\end{array} \right)   \]
}
\begin{framed}
\begin{verbatim}
A = [4, 7, 5
1,2,5
6,1,3]
\end{verbatim}
\end{framed}
\subsection*{Question C.1}
\begin{enumerate}
\item The inverse of matrix A ($A^{-1}$),
\item The rank of matrix $A$,
\item The product of A and B ($A \times B$).
\end{enumerate}




\subsection*{Question C.3}
Suppose we have the $3 \times 4$ matrix A.

\[ 
A = \left(
\begin{array}{cccc}
 5 & 2 & 1 & -1 \\ 
 1 & 3 & 2 & 5 \\ 
 -1 & 4 & 7 & 1 \\ 
\end{array} \right)
\]

\begin{enumerate}
\item Compute the sum totals for each column,
\item Compute the sum totals for each row,
\item Compute the overall sum total.
\end{enumerate}

\subsection*{Question C.4}

\[C = \left( \begin{array}{cccc}
2& 7& -5& 5\\
4&-2 & 7& -1\\
6&-3 & 0& 3 \\
4&-2 & 7& 1
\end{array} \right)
\]
\begin{framed}
\begin{verbatim}
C = [2, 7, -5, 5
4,-2,7,-1
6,-3,0,3
4,-2,7,1]

\end{verbatim}
\end{framed}
\begin{enumerate}
\item Replace all negative values of matrix C with 0. Call this matrix $C1$
\item Replace all negative values of matrix C with the absolute value of those values. Call this matrix $C2$.
\end{enumerate}

\end{document}
