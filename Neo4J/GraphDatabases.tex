\newpage
\section{Graph Analysis}
Graph analysis is among the hottest techniques around for making sense of large datasets, primarily by determining how tightly different data points are related or how similar they are. The term “graph” came into the broader lexicon along with social networks, which built social graphs to assess the relationships among their millions of users, but the technique has much broader uses.
\section{Graph Database}

A graph database is a database that uses graph structures with nodes, edges, and properties to represent and store data. By definition, a graph database is any storage system that provides index-free adjacency. This means that every element contains a direct pointer to its adjacent element and no index lookups are necessary. General graph databases that can store any graph are distinct from specialized graph databases such as triplestores and network databases.
Graph databases are based on graph theory. Graph databases employ nodes, properties, and edges. Nodes are very similar in nature to the objects that object-oriented programmers will be familiar with.

\end{document}
