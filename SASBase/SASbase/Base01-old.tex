-------------------------------------------%
\section{SAS Base 1: Basic Concept}
% Chapter 1
Descriptor Portion (p13)\\
Data Portion (p13)\\
Features of a SAS Window (p19)\\
Issuing Commands (p20)\\
Using Pop-Up Menus (p21)\\
Getting Help (p21)\\
Customizing Your SAS Enviroment (p22)\\
The Explorer Window (p22)\\
Clearing the Editor (p24)\\
Clearing the Editor (p25)\\
The Log Window (p25)\\
The output Windows (p25)\\
The Results window (p26)\\

Creating SAS files (p27)
\begin{verbatim}
·         Sashelp
·         Sasuser
·         Work
\end{verbatim}
Referencing SAS files (p31)
SAS Libraries
What is a SAS Library? A collection of SAS files, such as SAS data sets or catalogues.

\begin{verbatim}
Chapter Summary
·         Components of SAS Programs
·         Characteristics of SAS Programs
·         Processing SAS Programs
·         SAS Libraries
·         Referencing SAS files
·         SAS Data Sets
·         Variable Attributes
·         Using the main SAS Windows
\end{verbatim}




SAS Programs
SAS Libraries
Referencing SAS Files
SAS Data Sets
Using the Programming Workspace

Referencing SAS Files
Two-Levels Names
Referencing Temporary SAS Files
Referencing Permanent SAS Files
Rules for SAS Names
SAS Data Sets
Overview of Data Sets (p12)\\
Descriptor Portions (p13)\\
Data Portion (p13)\\
Observations(Rows) (p13)\\
Variables (Columns) (p14)\\
Missing Values (p14)\\
Variable Attributed (p14)\\
Name (p15)\\
Type (p15)\\
Length (p16)\\
Format (p16)\\
Informat (p17)\\
Label (p18)\\

Using the Programming Workspace
Using the Main SAS Package
Features of SAS Windows
Minimizing and Restoring Windows
Docking and Undocking Windows
Issuing Commands
Using Pop-Up Menues
Getting Help
