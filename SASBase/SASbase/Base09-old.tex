%------------------------------------------------------------------------------------------------------------------%
\section{SAS Base 9: Producing Descriptive Statistics}


Computing Statistics using PROC MEANS\\
Selecting Statistics\\
Limiting Decimal Places\\
Specifying Variables in PROC MEANS\\
Group Processing using the CLASS Statement\\
Group Processing using the BY Statement\\

\begin{itemize}
\item PROC SUMMARY
\item PROC MEANS
\end{itemize}

\subsection{\texttt{PROC MEANS}}
The default statistics produced by PROC MEANS are n-count,mean, maximum,minimum and standard deviation.

To specify variables to be analysed and included in the output, add a VAR statement and alist of the variable names
VAR varname1 varname 2;

\subsection{\texttt{PROC FREQ}}
By default, PROC FREQ creates a table of frequencies and percentages for both numeric and character variables.
Frequency distributions work best with variables that contain categorical variables.

Specifying Variables in PROC FREQ\\
Creating Two-Way Tables\\
Determining the Table Layout\\
Surpressing Table Information\\


(p271)
\begin{framed}
\begin{verbatim}
proc means data=clinic.diabetes min max maxdec=0
    var age height weight;
run;
\end{verbatim}
\end{framed}





(p283)
\begin{framed}
\begin{verbatim}
proc format
value wtfmt low ='<<140'
140-180='140-180'
181-high='>180';
run;
proc freq data =clinic.diabets
tables sex*weight / nofreq norow nocol;
format weight wtfml.;
run;
\end{verbatim}
\end{framed}


Computing statistics using PROC Means (p267)\\
Descriptive Statistics (p269)\\
Creating Two-Way Tables (p279)\\

