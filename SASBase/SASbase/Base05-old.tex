%------------------------------------------------------------------------------------------------------------------%
\section{SAS Base 5: Creating SAS Data Sets from Raw Files}
% SAS Base 5

Raw Data Files (p137)\\
A raw data file is an external file whose records contain data values that are organized in fields. Raw data fields are non-proprietary and can be read by a variety of programs.\\
Referencing a SAS library (p139)\\
Reference a SAS Library\\
LIBNAME Statement\\
Reference an external File\\
FILENAME statement\\
Using a LIBNAME statement
%------------------%
\begin{framed}
\begin{verbatim}
..
libname taxes ‘C:/Directory’;
..
\end{verbatim}
\end{framed}
%------------------%

Referencing a Raw Data File (p140)\\
Defining a fully qualified Filename (p141)\\
Defining an Aggregate Storage Location (p141)\\
Referencing a Fully Qualified Filename (p141)\\
Writing a Data Step Program (p142)\\
Naming the Data Step (p142)\\
Specifying the Raw data Film (p143)\\
Column Input (p144)\\
Standard and Nonstandard Numeric Data (p144)\\
Fixed-Field Data (p144)\\
Specifying Variable Names (p146)\\
SAS Naming Conventions\\
Must be 1 to 32 characters in length\\
Must be begin with a letter or underscore\\
Can continue with any combination of numbers, letters or underscores\\
Raw data can be organized in several different ways\\
Submitting the DATA Step (p147)\\
Verifying the Data Step (p147)\\
%------------------%
\begin{framed}
\begin{verbatim}
data sas.user.stress;
 	infile tests obs=10;
     input ID 1-4 Name $ 6-25
       	RestHR 27-29 MaxHR 31-33
       	RecHR 35-37 TimeMin 39-40
       	TimeSec 42-43 Tolerance $ 45;
run;
\end{verbatim}
\end{framed}
%------------------%

Listing the Data Set (p148)\\
Reading the Entire Raw Data filme (p149)\\
Invalid Data (p150)\\
Creating and Modifying Variables (p151)\\
SAS Expressions (p152)\\
Using Operators in SAS Expressions (p152)\\
Date Constants (p154)\\
Subsetting Data (p155)\\
Reading Instream Data (p156)\\
Creating a Raw Data File (p157)\\
Reading in Windows Excel Data (p161)\\
Referencing an Excel Workbook (p163)\\
The IMPORT wizard (p178)\\
Raw Data Files\\
Referencing a raw data file\\
Writing a DATA step program\\
Creating and modifying variables\\
Subsetting data\\
Reading instream data\\
Reading Microsoft Excel da\\
LIBNAME statement options\\

%SAS Base 5 p153
%------------------%
\begin{framed}
\begin{verbatim}
data sas.user stress;
infile tests;
input ID 1-4 Name $ 6-25 $ RehsHR
Tolerance $ 45;
TotalTime = (timemin*60)+timeset;
run;
\end{verbatim}
\end{framed}
%------------------%


p155



%------------------%
\begin{framed}
\begin{verbatim}
data sas.user stress;
infile tests;
..
if tolerance='D';
TotalTime = (timemin*60)+timeset;
run;
\end{verbatim}
\end{framed}
%------------------%



(p176)
%------------------%
\begin{framed}
\begin{verbatim}
libname clinic 'c:/bethesda/patients\admit';
 filname admit 'c:/bethesda/patients\admit';
data clinic.admittan;
 infile admit obs=5;
 input ID 1-4 Name $6-25 RestHR 27-29 MAXhr 31-33
 if tolerance ='D';
 TotalTime=(timemin*60) + timesec;
run;
proc print data=clinic.admittan;
run;
\end{verbatim}
\end{framed}
%------------------%




LIBNAME and FILENAME statements are global. librefs and filerefs remain ine ffects
until you change them, cancel then or end o SAS session.
For each field of raw data then you read into your data set, you must specify
the following in the INPUT statement: a valid SAS staement
a starting column and if necessary column,.
Column input is appropriate only in some
situations.
