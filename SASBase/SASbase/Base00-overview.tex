
SAS Base Certification Overview

\begin{itemize}
\item[A] Accessing Data
\item[B] Creating Data Structures
\item[C] Managing Data
\item[D] Generating Reports
\item[E] Handling Errors
\end{itemize}

http://www2.sas.com/proceedings/sugi22/BEGTUTOR/PAPER56.PDF

http://www.stattutorials.com/SAS/

http://itc.virginia.edu/research/sas/training/v8/

http://www.ats.ucla.edu/stat/sas/topics/graphics.htm

http://www.aaegrad.uga.edu/stata_sas_guide.pdf

http://www.ats.ucla.edu/stat/sas/glimmix.pdf


%============================================================%

1. Basic Concepts
2. Referencing Files and Setting Options
3. Editing and Debugging SAS programs
4. Creating List Reports

10. Producing HTML Output
11. Creating and Managing Variables
12. Reading SAS Data Sets
16. Processng Variables with Arrays
17. Reading Raw Data in Fixed Fields
18. Reading Free Format Data
19. Reading Data and Time Values
20. Creating a Single Observation from Multiple Records
21. Creating Multiple Observation from a single Record
22. Reading Hierarchical Files

<hr>

**A. Accessing Data**
- Use FORMATTED and LIST input to read raw data files.
- Use INFILE statement options to control processing when reading raw data files.
- Use various components of an INPUT statement to process raw data files including column and line pointer controls, and trailing @ controls.
- Combine SAS data sets.
- Access an Excel workbook.

**B. Creating Data Structures**
- Create temporary and permanent SAS data sets.
- Create and manipulate SAS date values.
- Export data to create standard and comma-delimited raw data files.
- Control which observations and variables in a SAS data set are processed and output.

**C. Managing Data**
- Investigate SAS data libraries using base SAS utility procedures.
- Sort observations in a SAS data set.
- Conditionally execute SAS statements.
- Use assignment statements in the DATA step.
- Modify variable attributes using options and statements in the DATA step.
- Accumulate sub-totals and totals using DATA step statements.
- Use SAS functions to manipulate character data, numeric data, and SAS date values.
- Use SAS functions to convert character data to numeric and vice versa.
- Process data using DO LOOPS.
- Process data using SAS arrays.
- Validate and clean data.

**D. Generating Reports**
- Generate list reports using the PRINT procedure.
- Generate summary reports and frequency tables using base SAS procedures.
- Enhance reports through the use of user-defined formats, titles, footnotes and SAS System reporting.
- Generate reports using ODS statements.

**E. Handling Errors**
- Identify and resolve programming logic errors.
- Recognize and correct syntax errors.
- Examine and resolve data errors.
