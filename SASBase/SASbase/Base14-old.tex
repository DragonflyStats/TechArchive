\section{Chapter 14 Transforming Data with SAS Functions}
Understanding SAS Functions (p415)\\
Uses of SAS Functions (p416)\\
General form of SAS Functions (p417)\\
Target Variable (p418)\\
Converting Data with Functions (p418)\\
Introduction to Converting Data (p418)\\
\begin{framed}
\begin{verbatim}
data hrd.newtemp;
    	set hrd.temp;
    	Salary=payrate*hours;
run;
\end{verbatim}
\end{framed}
%---------------------%
\begin{verbatim}
Potential Problems of Omitting INPUT or PUT (p419)
Automatic Character-to-Numeric Conversion (p419)
When Automatic Conversion Occurs (p420)
Restriction for WHERE conversion (p421)
Explicit Character-to Numeric Conversion (p421)
Automatic Number to character conversion (p423)
Matching the Data Type (p426)
Manipulating SAS Date Values with Functions (p427)
SAS Date Functions (p428)
SAS stores dates, times and date times as numeric values.
Function
Typical Use
MDY
TODAY
DATE
TIME
\end{verbatim}
%------------------%
\begin{framed}
\begin{verbatim}
date=mdy(mon,day,yr);
now=today();
now=date();
curtime=time()
\end{verbatim}
\end{framed}
%------------------%
YEAR,QTR,MONTH and DAY Function (p429)\\
Weekday Function (p432)\\
MDY Function (p433)\\
SAS’s Month, Day and Year functions\\
DATE and TODAY Functions (p436)\\
INTCK Function (p437)\\
INTNX Function (p438)\\
\begin{verbatim}
·         DAY
·         WEEKDAY
·         WEEK
·         TENDAY

·         SEMIMONTH
·         MONTH
·         QTR
·         SEMIYEAR
·         YEAR

DATDIF and YRDIF Functions (p440)
Introduction to Modifying Character values (p441)
Character Functions (p442)

·         SCAN
·         SUBSTR
·         TRIN
·         CATX
·         INDEX
·         FIND
·         UPCASE
·         LOWCASE
·         PROPCASE
·         TRANWRD


Scan Function (p442)
Specifying Delimiters (p443)
Specifying Multiple Delimiters (p443)
Default Delimiters (p443)
SCAN Function Syntax (p444)
\end{verbatim}
\begin{framed}
\begin{verbatim}
data hrd.newtemp(drop=name);
   set hrd.temp;
   LastName = scan(name,1);
   FirstName=scan(name,2);
   MiddleName=scan(name,3);
run;
\end{verbatim}
\end{framed}

Specifying the Variable Length (p445)\\
SCAN vs SUBSTR (p445)\\
SUBSTR (p445)\\
Replacing text using SUBSTR (p447)\\
Positioning the SUBSTR Function (p447)\\
The TRIM function (p449)\\
The CATX Function (p451)\\
This function allows you to concatenate character strings.\\
INDEX Function (p452)\\
Finding a string regardless of case (p454)\\
FIND function (p454)\\
LOWCASE Function (p456)\\
PROPCASE Function (p457)\\
TRANWRD Function (p458)\\
This function replaces or removes all occurrences of a pattern of characters within a character string.The translated characters can be located anywhere in the string.\\
ROUND Function (p460)\\
This function rounds values to the nearest specified unit.\\
Nesting SAS functions (p461)\\
Chapter Summary (p462)\\


SAS Base 14
Use of SAS Functions
-create sample statistics
-create SAS data values
-convert U.S. ZIP codes to state postal codes
-round values
-generate random numbers
-extract a portion of the character value
-covert data from one type to another
General Forms of SAS Functions
Arguments Variables List s and Arrays
Coverting Data with Functions
Introduction
Potential Problems
Automatic Character to Numeric Conversion\\
When Automatic Conversion Occurs\\
Restriction of WHERE Expressions\\
Explicit Character-to-Numeric Comvervsion\\
Automatic Numeric-to-Character Conversion\\
Explicit Numeric-to-Character Conversion\\
Matching the Data Type\\
WEEKDAY Function\\
MDY Function\\
DATA and TODAY Function\\
INTCK Function\\
INTNX Function\\
DATDIF and YRDIF Function\\
(p460)
\begin{framed}
\begin{verbatim}
data work.after;
 set work.before;
 Examples = int(examples);
run;
\end{verbatim}
\end{framed}

(p461)

\begin{framed}
\begin{verbatim}
data work.after;
 set work.before;
 Examples = int(examples,.2);
run;
\end{verbatim}
\end{framed}
