\section{SAS Base 19: Reading Date and Time Values}

Numeric SAS Day Values: Number of Days since  Juanuary 1 1960
SAS time values are the number of seconds since midnight,
Informats
\begin{itemize}
\item DATEw.
\item DATETIMEw.
\item MMDDYYw.
\item TIMEw.
\end{itemize}
TIMEw. reads time such as 17:00
Using Dates and Times in Calculations
Using Time and Date formats

\subsubsection{Chapter 19}
\begin{verbatim}
Modifying List Input (p560)
Using the & Modifier with a LENGTH Statement (p561)
Reading Nonstandard Values (p562)
Creating Free-Format Data (p565)
Specifying a Delimiter (p566)
Using DSD Options (p566)
Mixing Inputs Styles (p569)
\end{verbatim}
%-------------------%
\begin{framed}
\begin{verbatim}
options yearcutoff =1920;
data perm.aprbills;
   infile aprdata;
\end{verbatim}
\end{framed}
%------------------------------------------------------------------- 20 --%
