
                                                SAS

SAS is the leader in business analytics software and services, and the largest independent vendor in the business intelligence market and has been since 1976. SAS is used at more than 50,000 sites in over 100 countries, including 93 of the top 100 companies on the 2010 FORTUNE Global 500  list. SAS has an impact on everyone, every day. From the roads you travel and mortgage loans you purchase, to the brand of cereal you eat and cell phone plan you select, SAS plays a role in your daily life. SAS has helped organizations across all industries realize the full potential of their greatest asset: data. Simply put, SAS allows you to transform data about customers, performance, financials and more into information and predictive insight that lays the groundwork for solid and coherent decisions. 

More than 11,000 SAS employees in more than 50 countries and 400 SAS offices provide local, hands-on support for customers undergoing global software implementations. Technical and educational resources, reference information, consulting and worldwide SAS-specific groups are the backbone of business and technical users communities.

SAS offers an environment that inspires employees, allowing them to build long-term relationships with customers and focus on solving their business problems. In the face of economic uncertainty and market consolidation, SAS continues its 35-year history of growth and profitability. SAS customers use SAS for everything from developing drugs to efficiently marketing their products – and contribute to a lively, active SAS community worldwide. 


SAS constantly strengthens its products. Continued revenue growth allows SAS to reinvest a substantial percent of its revenues in R&D each year, resulting in expanded offerings related to cloud computing, unstructured data and grid computing, for example. This commitment to innovation is one reason an overwhelming majority of customers renew their SAS software licenses every year.

Customers even influence how SAS products evolve through the annual SASware Ballot that seeks user suggestions on new features they would like to see added to software and services.

SAS believes that a healthy workplace environment is critical for employees and for the business. Focusing on people and relationships leads to more productive, satisfied and dedicated employees. A healthy work-life balance is critical to this success.

As a result of this commitment, SAS US was named No. 1 on the FORTUNE 100 Best Companies to Work For list for 2010 and 2011. SAS Belgium, SAS Norway and SAS Sweden were all named No. 1 in their countries on the Best Place to Work For list by the Great Place to Work Institute. SAS offices in nine other countries have also been recognized for their workplace cultures.
SAS is also committed to corporate social responsibility, as evidenced in the environmental, employee and education efforts the company strives for year after year.

SAS 9.2 is the current release of SAS software. It provides the core components of the SAS Business Analytics Framework, enabling you to meet evolving business needs with the latest in SAS technology and analytics.

SAS 9.2 includes significant new analytic capabilities, new features for data integration, reporting enhancements, and improved OLAP storage. The release incorporates significant performance improvements, as well as tools that help IT more easily deploy, manage, and scale the SAS environment

Here’s an example of what SAS would look lik on a computer: 

 

There are many SAS programmers based in Ireland also as there is a wide demand for people of such skill. For example when I researched what kind of skills and responsibilities a person in Ireland with SAS programming qualifications the results were as follows:
• Development and delivery of loading external data
• Development and delivery of observational datasets to Sponsors and Biostatistics. 
• Development and delivery of CDISC compliant and sponsor defined data structures.
• Development and delivery of sponsor defined data management listings/reports
• Contributes to the creation of study level documentation and files according to departmental policy and regulatory requirements.
• Liaison with Data Management and Biostatistics on database specifications, timelines and quality requirements.

As well as this the job required the following qualifications:
• 2-3 years experience
• Appropriate computer-related qualification or equivalent
• Understanding of data structures is required as well as a basic understanding of the development and use of standard programs.
• Must work independently as well as part of a team and have good problem solving skills, attention to detail, verbal and written communications skills.



                                                       STATA

For over twenty-five years, StataCorp, located in College Station, Texas, has been a leader in statistical software, primarily through their flagship product Stata. Stata provides an integrated statistics, graphics, and data-management solution for anyone who analyzes data. 

StataCorp develops, distributes, and supports Stata software. They also publish books on Stata and statistics and publish a peer-reviewed quarterly journal through their publishing arm, Stata Press. 
Although there are no Stata distributers based in Ireland the nearest one is Timberlake Consultants Ltd and is based in London. Distributors and resellers offer prompt, reliable service for Stata sales and inquiries. Distributors provide support for basic technical questions, and typically offer training and consulting, or similar services. Stata distributors also carry Stata inventory, so delivery is fast. 

The Stata Journal is a quarterly publication containing articles about statistics, data analysis, teaching methods, and effective use of Stata's language. The Journal publishes reviewed papers together with shorter notes and comments, regular columns, book reviews, and other material of interest to researchers applying statistics in a variety of disciplines. 

Stata puts hundreds of statistical tools at your fingertips, from advanced techniques, such as survival models with frailty, dynamic panel data (DPD) regressions, generalized estimating equations (GEE), multilevel mixed models, models with sample selection, multiple imputation, ARCH, and estimation with complex survey samples; to standard methods, such as linear and generalized linear models (GLM), regressions with count or binary outcomes, ANOVA/MANOVA, ARIMA, cluster analysis, standardization of rates, case–control analysis, and basic tabulations and summary statistics. 

Stata’s data-management commands give you complete control of all types of data: you can combine and reshape datasets, manage variables, and collect statistics across groups or replicates. You can work with byte, integer, long, floats, double, and string variables. Stata also has advanced tools for managing specialized data such as survival/duration data, time-series data, panel/longitudinal data, categorical data, multiple-imputation data, and survey data. 

Stata makes it easy to generate publication-quality, distinctly styled graphs, including regression fit graphs, distributional plots, time-series graphs, and survival plots. With the integrated Graph Editor you click to change anything about your graph or to add titles, notes, lines, arrows, and text. 

Here are some examples of STATA graphs:

 

 

 







