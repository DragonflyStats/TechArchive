




--------------------------------------------------------------
Chapter 17

#### Reading Non Standard Numeric Data (page 535)

- blanks, commas, dashes, dollar signs, percent signs, right parentheses, left parentheses


-------------------------------------------------------------

Chapter 19


- How SAS stores Data Values
- How SAS stores Time Values
- More about SAS Time and Date Values




------------------------------------------------------------
Chapter 21


<pre><code>
data perm.sales97;
   infile data97 missover;
   input ID $ sales : comma. @;
   Quarter=0
do while (sales ne .);
   quarter+1;
   output; 
   input sales : comma. @;
end;
run;   


</code></pre>

------------------------------------------------------------

Chapter 22


-----------------------------------------------------------------------------

Informats and Formats

An informat is a specification for how raw data should be read.

A format is a layout specification for how a variable should be printed or displayed.

SAS contains many internal formats and informats, or user defined formats and informats can be constructed using PROC FORMAT.

---------------------------------------------

Conditional Processing

http://www.puzha.com/sasbook/working%20with%20sas%20datasets%20part3.html

---------------------------------------------


http://www.ssc.wisc.edu/sscc/pubs/4-18.htm


Proc Print

 

Proc print simply prints the contents of a data set to the lst file. The basic syntax is

 

proc print data=dataset;

 run;

 

All you have to do is specify the data set.

 

By default SAS will format the output such that it will be centered on a printed page.

You can override this behavior by adding

 

options nocenter;

 

right before the proc print. This can make it easier to read on the screen.

You can also set the line size and page size SAS will use.



referencing a SAS library

referencing a data file

name a SAS data set to be created

specifying a raw data file to be read

creating new variables and assigning values


reading in instream data

submittting and verifying a DATA step program



P95 P50 -  percentiles

Q1,Q3 - Quartiles

--------------------------------------------------------------------------------



 


