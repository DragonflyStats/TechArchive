
\documentclass[12pt, a4paper]{report}
\usepackage{natbib}
\usepackage{vmargin}
\usepackage{graphicx}
\usepackage{epsfig}
\usepackage{subfigure}
%\usepackage{amscd}
\usepackage{amssymb}
\usepackage{subfigure}
\usepackage{amsbsy}
\usepackage{amsthm, amsmath}
%\usepackage[dvips]{graphicx}
\bibliographystyle{chicago}
\renewcommand{\baselinestretch}{1.1}

% left top textwidth textheight headheight % headsep footheight footskip
\setmargins{2.0cm}{2cm}{15.5 cm}{23.5cm}{0.5cm}{0cm}{1cm}{1cm}

\pagenumbering{arabic}


\begin{document}
\author{Kevin O'Brien}
\title{Advanced SAS}
\date{\today}
\maketitle

\tableofcontents \setcounter{tocdepth}{2}

\newpage
%-----------------------------------------------------------------------------------------%
\chapter{SAS SQL}
\section{Table Joins}

\begin{itemize}
\item Inner Joins
    \begin{itemize}\item An inner join is the most
    common join operation used in applications and can be regarded as
    the default join-type. \item Inner join creates a new result table
    by combining column values of two tables (A and B) based upon the
    join-predicate.
    \end{itemize}
\item Right Joins

\item Left Joins

\item Full Joins
    \begin{itemize}
    \item A full join combines the results of both left and right
    outer joins. \item The joined table will contain all records from
    both tables, and fill in NULLs for missing matches on either side.
    \end{itemize}
\end{itemize}


%----------------------------------------------------------------------------------------Chapter 2%
\chapter{SAS Macro Language}
\section{Macros}
SAS macro language gives the opportunity to insert programming
steps inside a SAS code. The first advantage is to simplify large
programs. Indeed, it is possible to create functions with
parameters, which can be easily invoked several times. It is
therefore possible to define a model and submit it to statistical
testing without having to use predefined functions. All the usual
programming elements: if then else statements, loops for I=1 to N
do, and other similar operators can be used to this effect.


\begin{itemize}
\item The most common macro statements are $\%MACRO$ and $\%MEND$.
\item Other macro statements, often similar to their regular SAS
equivalents, also start with the $\%$ symbol, such as $\%DO$,
$\%END$, $\%IF\%THEN$, $\%ELSE$, $\%GLOBAL$, $\%LOCAL$, and
$\%LET$. \item Macro statements are ended with semicolons (;).
\end{itemize}

\section{Macro Variables}
Macro variables are indicated by preceding a variable name with an
ampersand ($\&$). To define a macro variable use the $\%LET$
statement. For example to assign the value 24 to the variable
named `sto':
\begin{verbatim}
%LET sto=24;
\end{verbatim}

Macro variables that contain entire sections of a SAS program can
also be created:
\begin{verbatim}
%let sto2=%str(proc means data=year;
var rainfall; run; );
\end{verbatim}

\section{Example of code}
\begin{verbatim}%MACRO CLAIMREP;
%LET CURYEAR = 98;
%DO YEAR = 89 %TO 98;
%IF “&YEAR” = “&CURYEAR” %THEN %LET
NOTE = FOOTNOTE “THRU &SYSDATE”; PROC PRINT DATA = CLAIMS; WHERE
YEAR = &YEAR; TITLE1 “ANNUAL REPORT FOR 19&YEAR”; &NOTE;
%END;
%MEND CLAIMREP;
%CLAIMREP;
\end{verbatim}


\section{Macro functions}
Macro functions operate much as regular SAS functions, except that
the arguments are within the context of the macro language. Most
operate on character strings (e.g., $\%SUBSTR$) or control the
exact interpretation of macro special symbols (e.g., $\%STR$).
Macro variables are character variables so the $\%EVAL$ function
may be required to perform arithmetic within macro statements.

\section{Debugging Macros}
Identifying and fixing macro language errors can be quite a
challenge. Good programming practice will help to limit the extent
of the task. This includes designing the macro program carefully.
Test regular programming steps to confirm that they work (and are
not a source of apparent macro errors). Then build and test the
macros incrementally, commenting out macro statements irrelevant
to the current testing.
\end{document}
