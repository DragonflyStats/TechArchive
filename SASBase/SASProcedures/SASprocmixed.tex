SAS Mixed Models

Introduction to Mixed Models
identifying fixed and random effects
describing linear mixed model equations and assumptions
fitting a linear mixed model for a randomized complete block design using the MIXED procedure
writing CONTRAST and ESTIMATE statements to perform custom hypothesis tests
Examples of Mixed Models in Some Designed Experiments
fitting a linear mixed model for two-way mixed models
fitting a linear mixed model for nested mixed models
fitting a linear mixed model for split-plot designs
fitting a linear mixed model for crossover designs
Examples of Mixed Models with Covariates
fitting analysis of covariance models with random effects
performing random coefficient regression analysis
conducting hierarchical linear modeling
Best Linear Unbiased Prediction
explaining BLUPs and EBLUPs
producing parameter estimates associated with the fixed effects and random effects
explaining the difference between LSMEANS and EBLUPs
computing LSMEANS and EBLUPs using the MIXED procedure
Repeated Measures Analysis
discussing issues on repeated measures analysis, including modeling covariance structure
analyzing repeated measures data using the four-step process with the MIXED procedure
Mixed Models Residual Diagnostics and Troubleshooting
performing residual and influence diagnostics for linear mixed models
troubleshooting convergence problems
Additional Information about Linear Mixed Models (Self-Study)
discussing issues associated with unbalanced data, data with empty cells, estimation and inference of variance parameters, and different denominator degrees of freedom estimation methods
Generalized Linear Mixed Models and Nonlinear Mixed Models
discussing the situations where generalized linear mixed models and nonlinear mixed models analysis are needed
performing the analysis for generalized linear mixed models using the GLIMMIX procedure

